\documentclass[letterpaper,twocolumn,10pt]{article}
\usepackage{epsfig,endnotes}
\usepackage{varwidth}
\usepackage{algorithm}
\usepackage{algpseudocode}
\usepackage{balance}
\usepackage{xcolor}
\usepackage{nicefrac}
\usepackage{amsmath}
\usepackage{braket}
\usepackage{bm}
\usepackage{mathtools}
\usepackage{multirow}
\usepackage{bigdelim}
\usepackage{mathtools}
\usepackage{amssymb}
\usepackage{indentfirst}
\usepackage{booktabs}
\usepackage{enumitem}
\usepackage{boxedminipage}
\usepackage{float}
\usepackage{graphicx}
\usepackage{caption}
\usepackage{cleveref}
\usepackage{subcaption}
\usepackage[normalem]{ulem}
\usepackage{xpatch}
%\usepackage{MnSymbol}
\usepackage{xspace}
\usepackage{listings}
\usepackage{mathpartir}

\usepackage[pdftex,bookmarks=true,pdfstartview=FitH,colorlinks,linkcolor=blue,filecolor=blue,citecolor=blue,urlcolor=blue,pagebackref=true]{hyperref}
    \urlstyle{sf}

\usepackage{stmaryrd}

\newlength{\saveparindent}
\setlength{\saveparindent}{\parindent}
\newlength{\saveparskip}
\setlength{\saveparskip}{\parskip}
 
 
\newenvironment{tiret}{%
\begin{list}{\hspace{2pt}\rule[0.5ex]{6pt}{1pt}\hfill}{\labelwidth=15pt%
\labelsep=5pt \leftmargin=20pt \topsep=3pt%
\setlength{\listparindent}{\saveparindent}%
\setlength{\parsep}{\saveparskip}%
\setlength{\itemsep}{0pt} }}{\end{list}}
 
\newenvironment{enum}{%
\begin{list}{{\rm (\arabic{ctr})}\hfill}{\usecounter{ctr}\labelwidth=17pt%
\labelsep=6pt \leftmargin=23pt \topsep=5pt%
\setlength{\listparindent}{\saveparindent}%
\setlength{\parsep}{\saveparskip}%
\setlength{\itemsep}{3pt} }}{\end{list}}
 
\newenvironment{newenum}{%
\begin{list}{{\rm \arabic{ctr}.}\hfill}{\usecounter{ctr}\labelwidth=17pt%
\labelsep=6pt \leftmargin=23pt \topsep=.5pt%
\setlength{\listparindent}{\saveparindent}%
\setlength{\parsep}{\saveparskip}%
\setlength{\itemsep}{5pt} }}{\end{list}}

\newcommand{\tool}{{\textsc{EzPC}}\xspace}
\newcommand{\minion}{{\textsc{MiniONN}}\xspace}
\newcommand{\bonsai}{{\textsc{Bonsai}}\xspace}
\newcommand{\R}{{\mathbb{R}}\xspace}
\newcommand{\mpc}{{MPC}\xspace}

\newtheorem{problem}{Problem}
\newtheorem{theorem}{Theorem}
\newtheorem{conjecture}[theorem]{Conjecture}
\newtheorem{definition}[theorem]{Definition}
\newtheorem{lemma}[theorem]{Lemma}
\newtheorem{proposition}[theorem]{Proposition}
\newtheorem{corollary}[theorem]{Corollary}
\newtheorem{claim}[theorem]{Claim}
\newtheorem{fact}[theorem]{Fact}
\newtheorem{remk}[theorem]{Remark}
\newtheorem{apdxlemma}{Lemma}


\newcommand{\namedref}[2]{\hyperref[#2]{#1~\ref*{#2}}\xspace}
\newcommand{\lemmaref}[1]{\namedref{Lemma}{lem:#1}}
\newcommand{\propref}[1]{\namedref{Proposition}{prop:#1}}
\newcommand{\theoremref}[1]{\namedref{Theorem}{thm:#1}}
\newcommand{\claimref}[1]{\namedref{Claim}{clm:#1}}
\newcommand{\corolref}[1]{\namedref{Corollary}{corol:#1}}
\newcommand{\figureref}[1]{\namedref{Figure}{fig:#1}}
\newcommand{\tableref}[1]{\namedref{Table}{tbl:#1}}
\newcommand{\equationref}[1]{\namedref{Equation}{eq:#1}}
\newcommand{\defref}[1]{\namedref{Definition}{def:#1}}
\newcommand{\observationref}[1]{\namedref{Observation}{obs:#1}}
\newcommand{\procedureref}[1]{\namedref{Procedure}{proc:#1}}
\newcommand{\importedtheoremref}[1]{\namedref{Imported Theorem}{impthm:#1}}
\newcommand{\informaltheoremref}[1]{\namedref{Informal Theorem}{infthm:#1}}

\newcommand{\sectionref}[1]{\namedref{Section}{sec:#1}}
\newcommand{\appendixref}[1]{\namedref{Appendix}{app:#1}}
\newcommand{\propertyref}[1]{\namedref{Property}{prop:#1}}

\newcommand{\algoref}[1]{\namedref{Algorithm}{algo:#1}}



\newcommand{\TODO}[1]{{{\color{red} TODO: #1}}}
\newcommand{\divya}[1]{{{\color{blue} dg: #1}}}
\newcommand{\cmmt}[1]{{{\color{red} check: #1}}}
\newcommand{\nc}[1]{{{\color{purple} nc: #1}}}
\newcommand{\rs}[1]{{{\color{magenta} rs: #1}}}
\newcommand{\aseem}[1]{{{\color{purple} Aseem: #1}}}
\newcommand{\adv}{\mathcal{A}}
\newcommand{\env}{\mathcal{Z}}
\newcommand{\prot}{\Pi}
\newcommand{\real}{\mathsf{REAL}}
\newcommand{\ideal}{\mathsf{IDEAL}}
\newcommand{\simu}{\mathcal{S}}
\newcommand{\F}{\mathcal{F}}
\newcommand{\secparam}{\kappa}




\lstset{ % 
    language=C,
    backgroundcolor=\color{white},   
    basicstyle=\footnotesize\ttfamily\bfseries,
    breakatwhitespace=false,
    breaklines=false,
    belowskip=-0.3cm,
    captionpos=b,                    
    commentstyle=\color{red},
    deletekeywords={...}, 
    escapeinside={\%*}{*)}, 
    extendedchars=true, 
    %% frame=single,
    keepspaces=true,
    keywordstyle=\color{blue},
    keywordstyle=[2]\color{blue},
    otherkeywords={*,...,in,uint,input1,input2,output},
    keywords=[2]{private},
    numbers=left,
    numbersep=5pt, 
    numberstyle=\tiny\color{gray}\bfseries, 
    rulecolor=\color{black},
    showspaces=false,
    showstringspaces=false, 
    showtabs=false, 
    stepnumber=2, 
    stringstyle=\color{mymauve},
    tabsize=2, 
    title=\lstname
}

\let\ls\lstinline


\usepackage{usenix,epsfig,endnotes,pdfpages}
\begin{document}

%don't want date printed
\date{}

\title{\tool: Programmable, Efficient, and Scalable Secure Two-Party Computation}


%for single author (just remove % characters)
%% \author{
%% {\rm Your N.\ Here}\\
%% Your Institution
%% \and
%% {\rm Second Name}\\
%% Second Institution
%% % copy the following lines to add more authors
%% % \and
%% % {\rm Name}\\
%% %Name Institution
%% } % end author


\maketitle

\subsection*{Abstract}

  %% Secure two-party computation (\mpc) allows two mutually distrusting
  %% parties to compute any function on their joint inputs, without
  %% revealing anything other than the output of the function to one
  %% another.

  %While the application and potential impact of secure computation is virtually limitless, writing an efficient and scalable secure protocol is by no means an easy feat.
% Prior work in designing such protocols, suffered from either a) a lack of programmability (e.g. requiring the programmer to write cryptography-aware circuits for the computation); b) a lack of efficiency for several natural computations (such as machine learning classification tasks); or c) required cryptographers to design hand-crafted protocols for various applications.
We present \tool: a 
%new ``cryptographic-cost aware'' 
secure two-party computation (\mpc) framework that generates efficient
\mpc protocols from high-level, easy-to-write, programs.
\tool provides formal correctness and security guarantees while maintaining performance and scalability.
%% that 
%% %contain absolutely no cryptographic details. Programs in our language 
%% are easy-to-write and are statically verified. 
%Furthermore,
Previous language frameworks, such as CBMC-GC, ObliVM, SMCL, and
Wysteria,
generate protocols that use either arithmetic or boolean
circuits exclusively. Our compiler is the first to generate protocols
that combine both arithmetic sharing and garbled circuits  for better
performance.
%We give the first compiler that generates protocols which combine both arithmetic and boolean circuits
%for better performance compared to protocols that use either arithmetic or boolean circuits exclusively.
We
empirically demonstrate that the
protocols generated by our framework match or outperform (up to
19x)
recent works that provide hand-crafted protocols for various
functionalities such as secure  prediction  and matrix factorization.

\aseem{Usenix TODOs:
  (a) Explain cryptographic cost-awareness better (Reviewer A, Rahul). -- left for future.

  (b) Comment about side-channels (Reviewer A, Aseem). -- will add in Usenix.

  (c) Remove the comment about extension to malicious security (Reviewer A, Nishanth/Divya)? \nc{I think we can keep it, but I've added a caveat that a more complete treatment of malicious security is left to future work.} -- done.

  (d) Highlight advantages of EzPC more in Section 2 (Reviewer A, Aseem). -- add heading to the S2 para.

  (e) Add proofs and defns to appendix, Usenix has no limit (Reviewer B, Aseem). -- will add for usenix.

  (f) More explanation to Section 4, with a small example (Reviewer B, Nishanth). \nc{Done.} -- will move to Appendix B.

  (g) Comment about high branches (branches on secrets), we plan to
  support proper declassification in future (Reviewer B, All)? -- ignore for now.

  (h) Can the compiler reject secure programs -- of course! Type
  systems are inherently conservative. May be we don't need to add
  anything about it (Reviewer B, All)? -- ignore for now.

  (i) How are optimizations secure (Reviewer B, Divya/Aseem). -- add in S5.

  (j) Compiler validation, read about Frigate and add to related work
  (Reviewer C, Aseem). -- probably add.

  (k) Discuss what would it take to handle malicious parties -- may be
  we don't address this (Reviewer C, All)? \nc{Added a little more to conclusions.}

  (l) Limitation of manual partitioning -- it is what it is at this
  point (Reviewer C, All). -- ignore for now.

  (m) Getting code from others, consistent machine configuration for
  Bost et al. (Reviewer C, Rahul/Nishanth). \nc{Wrote a line.}

  (n) Comment on why are we better than Minion (Reviewer C, Rahul). -- add a line, reword ``surprising''.
}
%\tool provides formal correctness and security guarantees.
%% (up to
%% a 19x factor!).


\graphicspath{{./Images/}}
\DeclareGraphicsExtensions{.pdf,.jpg,.png}

  
  
  %% \TODO{While taking a pass on intro, check citation correctness or if we are missing any citation.}
\vspace{-0.1in}
\section{Introduction}
\label{sec:intro}

%% \aseem{The intuition of this paragraph is conveyed in rest of the
%% intro already?}

 Today it is hard for developers to program secure applications
 using cryptographic techniques. Typical developers lack a deep
 understanding of cryptographic protocols, and cannot be
 expected to use them correctly and efficiently on their own.
 Ideally, a developer would declare the functionality in a general
 purpose, high-level programming language and a
 tool, e.g. a compiler, would generate an efficient protocol that
 implements the functionality securely, while hiding the cryptography
 behind-the-scenes. 
 
 This paper presents such a framework for Secure
 Two-party Computation (\mpc),
%
%Most cryptographic tools rely on circuit as abstractions for any general computation but real-life developers write programs. It is critical to address this fundamental  representation gap to make cryptography accessible to non-crypto specialist developers. In this work, we address this gap for the case of secure multi-party computation.
%
%Most cryptographic tools are very hard to use non-crypto specialist developers. The task becomes even more tedious and error-prone for the case of secure multiparty computation since different protocol implementations require the function to be expressed in different representations such as Boolean or Arithmetic circuits. To make cryptography accessible to non-crypto specialist developers, we need to design systems that do crypto automatically and satisfy the properties of ... 
%
%Secure Two-party Computation (\mpc)~(\cite{yao,gmw}) is
a powerful
cryptographic technique that allows two mutually distrusting parties
to compute a publicly known joint function of their secret inputs in a way that both
the parties learn nothing about the inputs of each other beyond what
is revealed by their (possibly different) outputs. For example, \mpc
can be used for {\em secure
prediction}~(\cite{shafindss,wu,barni,minionn,secureml}),
where one party (the server) holds a proprietary classifier to predict
a label (e.g., a disease, genomics, or spam detection), and the other
party (the client) holds a private input that it wants to run the
classifier on. Using \mpc guarantees that the server learns nothing
about the client's input or output, and that the client learns nothing
about the classifier, beyond what is revealed by the output
label.

%Since its introduction in 1980s, there has been a long
%sequence of work that has transformed 2PC from a mere theoretical tool to
%something practically efficient and usable.
%Our goal is to build a framework that enables programmers, who are not
%experts in cryptography, to easily program efficient and scalable \mpc
%protocols. 
To understand the
state-of-the-art, let us consider an example underlying many
secure prediction algorithms. Suppose Alice wants to write
a \mpc protocol to securely
compute $w^Tx >b$. Here $w$ (a vector) and $b$ (a scalar) constitute
the server classifier, and $x$ is the client's input vector. Further,
$\cdot^{T}$ is the matrix transpose operator, and $w^Tx$ denotes the
inner product of  $w^T$ and $x$. Alice has the following options.

%\begin{tiret}
%\item

She can program the computation in one of the several programmer
friendly, domain-specific languages (such
as Fairplay~\cite{fairplay}, Wysteria~\cite{wysteria},
ObliVM~\cite{oblivm}, CBMC-GC~\cite{cbmcgc}, SMCL~\cite{smcl},
Sharemind~\cite{sharemind}, \cite{lambdaps} etc.) that would
automatically compile it to a \mpc protocol. However, all of these
frameworks use cryptographic backends that take as input
the computation expressed either as a boolean
circuit~(\cite{yao,gmw}) or as an arithmetic circuit~(\cite{sss,viff,gentry}). 
%
The efficiency of the generated \mpc protocol is thus bounded by the
efficiency of representing the computation in \emph{one} of
these representations. For instance, multiplication of two
$\ell$-bit integers can either be expressed as a boolean circuit of
size $O(\ell^2)$, or as an arithmetic circuit with 1 multiplication
gate.
It is well-known that boolean circuits are not suitable for doing arithmetic operations such as integer multiplications but are unavoidable for boolean operations such as comparison \cite{aby,tasty,autoS,secureml,minionn,chameleon}.
%
For better efficiency, Alice would ideally like to compute
$w^Tx$ using an arithmetic circuit, and the comparison with $b$ using
a boolean circuit.
%

Unfortunately, none of the above frameworks support combinations of
arithmetic and boolean circuits, and using different tools for
different parts of the computation is cumbersome and error-prone.

%% In fact, most interesting functions require a mix of arithmetic and
%% boolean computations for efficiency.

%\item

Alternatively, Alice can use a tool such as ABY (Demmler et
al.~\cite{aby}) that allows
the computation to be expressed as a combination of arithmetic and
boolean circuits. However, here, the programming interface is quite
low-level: the programmer is required to first manually split the
computation into arithmetic and boolean components, and then write the
circuits for all the components manually, including the appropriate
inter-conversion gates between them. Clearly, writing correct
and efficient protocols in such a framework is beyond an average
programmer who does not understand the various trade-offs between
arithmetic and boolean circuits, and even for an expert cryptographer,
writing large computations in such a framework can be tedious (a sentiment echoed by Demmler et al.~\cite{aby} themselves).

%\item
A third option for Alice is to earn a PhD in cryptography, and design
and implement specialized, efficient \mpc protocols (similar
to~\cite{shafindss,wu,minionn,valeriaMatrix}) for her tasks.
%\end{tiret}
%In our work, we achieve the best of all the above options. 




%Unfortunately, implementing functionalities using \mpc protocols requires thorough understanding of cryptography. To allow for widespread use of \mpc, it is critical that \mpc protocols are programmable by non-cryptographic experts. To cater to this, there have been several efforts developing domain-specific languages that are programmer friendly and compile to a SMC protocol. Most notable of these domain specific languages and systems include \cite{...}. All of these works build on a cryptographic back-end that is either entirely boolean \cite{yao,gmw} or arithmetic \cite{homo}. In fact, most of these works use boolean circuits based scheme for completeness\footnote{Note that comparisons cannot be expressed in arithmetic circuits}. However, most interesting functions require a mix of arithmetic and boolean computations.  Examples include ......... As one can expect, compiling these programs to boolean circuits is one of the biggest efficiency bottlenecks. \divya{Our work addresses this performance bottleneck.}

%To address this issue \divya{on the cryptographic side}, recently Demmler et al. \cite{aby} gave a cryptographic protocol for SMC where the parties can mix arithmetic and boolean computations and claimed potentially great performance benefits. However, using their system requires the programmer to be aware of trade-offs of arithmetic and boolean cryptographic schemes. In their framework, the programmer is required to write circuits consisting of a mix of arithmetic and boolean gates along with appropriate conversion gates. In short, as is also mentioned by the authors themselves, the current system is not suitable to be used by non-specialist programmers.


%%%%%%%Recent works \cite{aby,secureml,minion} have shown that great performance benefit can be obtained by mixing arithmetic and boolean computations in the back-end. Certain computations such as integer multiplication are efficient when implemented using arithmetic SMC \cite{gmw} whereas max(x,0) needs to be computed using a boolean back-end \cite{yao}. More details later.. ABY \cite{aby} requires programmer to be aware of arithmetic and boolean trade-offs and  write high-level circuits consisting of both arithmetic and boolean gates and share-conversion gates. Other works such as \cite{secureml,minion} build on ideas from \cite{aby} and develop tailor made algorithms for neural network training and prediction \cite{ml} and claim huge improvements over only boolean implementations. As is already mentioned in these works, their systems are mere proofs-of-concept and far from being implementable.



%\divya{old para...} In this work, we develop and implement our framework \tool\footnote{\tool, read as ``easy peasy'', stands for Easy 2 Party Computation} for secure
%computation that achieves generality, performance and programmer
%productivity. The programmer writes a high level \divya{C-like}
%program (instead of a circuit) to describe the function to be
%computed. Our compiler automatically generates an \mpc protocol using a
%mix of Arithmetic and Boolean compute. Our compiler is general and can
%work with any secure implementation of a mix of arithmetic and boolean
%computations. In our work, we focus on semi-honest secure two-party
%computation. We build on ABY \cite{aby} that provides the suitable
%cryptographic back-end in this setting. We provide a formal type
%system and prove correctness and security of our compiler. We evaluate
%our system on various benchmarks such as logistic regression and
%convolutional neural network (CNN) for MNIST data \cite{minionn},
%naive bayes, decision trees from \cite{shafindss}. Evaluations show
%that protocols generated by our compiler match
%the performance or outperform hand-crafted protocols in most cases (see
%\sectionref{eval}). Below, we will  give an overview of \tool and
%describe our contributions in more detail.

%The programmer is oblivious of the cryptographic back-end being used. Our compiler automatically compiles it to a circuit framework consisting to both arithmetic and boolean gates as well share-conversion gates wherever required. We use ABY as the cryptographic backend. \divya{Say something about what kind of backend we want. provides \mpc for a mix of arithmetic and boolean circuit with appropriate secure conversion between two types and ABY provides such a framework for semi-honest 2pc.} Our work is compatible with malicious, multiparty as well... etc We give a type system, correctness ..... We evaluate our framework on ... and show performance comparable and even better than tailor made protocols. It is simple to program in our framework and our work provides a meaningful baseline for future works designing tailor made protocols for specific functionalities.

%\subsubsection*{\tool}
%\label{sec:contrib}
%Write high level overview of programming language and tool.. \divya{have some preamble}

This paper presents \tool\footnote{Read as ``easy peasy'',
stands for Easy 2 Party Computation.}, the first
``cryptographic-cost aware'' compiler that generates efficient and
scalable \mpc protocols using combinations of
arithmetic and boolean circuits. \tool is backed by a formal model
that enables it to choose arithmetic or boolean representations for
different parts of
the program, while automatically inserting inter-conversion gates as
necessary. In addition to guiding the compiler, the formal model also
provides strong correctness and
security theorems. Our comprehensive evaluation shows that the
automatically generated protocols have performance comparable to
or better than the custom, specialized
protocols from previous works~\cite{shafindss,wu,minionn,secureml,cryptonets,valeriaMatrix}.
In fact, these papers (and others) cite the inefficiency of generic \mpc as the major motivation
behind the design of specialized protocols. Using \tool, we empirically  demonstrate
that generic \mpc implementations are much more efficient than what they were believed to be. 
Below we describe
the salient features of \tool.


%% that introduce unique compilation challenges.
%% Standard compilers do not attempt to reason about or insert non-trivial type casts
%% and require the programmers to perform them explicitly. \aseem{This
%% last sentence and the bit about unique compilation challenges seems
%% unnecessary to me. Compilers do insert type casts automatically. Also
%% at this point in the paper the words ``type casts'' don't mean much.}
%% We back \tool with formal semantics that enable the compiler to
%% insert type conversions between boolean and arithmetic components
%% while maintaining provable correctness, security, and
%% efficiency. \aseem{Provable efficiency? Perhaps provable should
%% be rescoped. And again, type conversions may not be clear at this
%% point. I would simply say that EzPC inserts inter-conversion gates
%% automatically. It is backed by a formal model and provides
%% provable correctness and security, and efficiency.}

%% This paper presents \tool\footnote{Read as ``easy peasy'',
%% stands for Easy 2 Party Computation.}, the first
%% ``cryptographic-cost aware'' framework that generates efficient and
%% scalable \mpc protocols from high-level programs devoid of any
%% cryptographic details. 
%% The generated protocols use combinations of
%% arithmetic and boolean circuits that introduce unique compilation challenges.
%% Standard compilers do not attempt to reason about or insert non-trivial type casts
%% and require the programmers to perform them explicitly. \aseem{This
%% last sentence and the bit about unique compilation challenges seems
%% unnecessary to me. Compilers do insert type casts automatically. Also
%% at this point in the paper the words ``type casts'' don't mean much.}
%% We back \tool with formal semantics that enable the compiler to
%% insert type conversions between boolean and arithmetic components
%% while maintaining provable correctness, security, and
%% efficiency. \aseem{Provable efficiency? Perhaps provable should
%% be rescoped. And again, type conversions may not be clear at this
%% point. I would simply say that EzPC inserts inter-conversion gates
%% automatically. It is backed by a formal model and provides
%% provable correctness and security, and efficiency.}
%% The automatically generated protocols have performance comparable to
%% or better than the custom, specialized
%% protocols from previous works~\cite{shafindss,wu,minionn,secureml,cryptonets,valeriaMatrix}.
%% In fact, these papers (and others) cite the inefficiency of generic \mpc as the major motivation
%% behind the design of specialized protocols. Using \tool, we empirically  demonstrate
%% that generic \mpc implementations are much more efficient than what they were believed to be. 
%%  %(up to a factor of 19x in our
%% %experiments, depending on the functionality) \nc{I added this but I
%% %  guess we are saying it many times, but might be nice to highlight
%% %  perf early no?} \aseem{My rationale was to save the numbers for
%% %  Evaluation summary and remain high-level here.} 
%% Below we describe
%% the salient features of \tool.

\begin{figure}
\begin{lstlisting}[language=C,mathescape=true]
$\mathtt{
uint\;w[30] = input1();\;\;uint\;b = input1();
}$
$\mathtt{
uint\;x[30] = input2();
}$
$\mathtt{
uint\;acc = 0;
}$
$\mathtt{
for\;i\;in\;[0:30]\;\{\;acc\;=\;acc\;+\;(w[i]\;\times\;x[i]);\;\}
}$
$\mathtt{
output2((acc > b)\;?\;1\;:\;0)\;\textbf{{\color{blue}//only to party 2}}
}$
\end{lstlisting}
\caption{\tool code for $w^Tx >b$}
\label{fig:ex-sml}
\end{figure}

\paragraph{Ease of programming.} \tool source programs are ideal
functionalities that describe ``what'' computation needs to be done,
rather than ``how'' to do it. In particular, the programmer writes the
high-level computation without thinking about the underlying
cryptographic details. For example, \figureref{ex-sml}
shows an \tool source
program for $w^Tx >b$. The program is quite similar to what a programmer might
write in C++ or Java. The simplicity of the
language comes with the usual benefits: it is easily accessible to the
developers, there are fewer avenues for making mistakes, developers
don't bear the burden of getting cryptographic details right, 
code optimizations can be left to the compiler,
and it
is easy to maintain and modify the programs. Needless to say,
frameworks that expose low-level circuit APIs to the programmer do not
enjoy these benefits.

\paragraph{Cryptographic-cost aware compiler.} The \tool compiler
compiles a source program to a hybrid computation consisting
of \emph{public} and \emph{secret} parts. In the example above, for
instance, \tool compiler realizes that the array index~\ls{i} is
public, and generates non-cryptographic code for the array accesses.
Further, within the secret parts, \tool compiler is
aware
of the cryptographic costs of arithmetic and boolean representations
of the source language operators. Based on these costs, the compiler
automatically picks arithmetic or boolean
representations for different sub-parts, and generates the
corresponding circuits along with the required inter-conversion
gates. The outcome is an efficient \mpc protocol combining arithmetic
and boolean circuits, while the programmer remains
oblivious of all these cryptographic details. Indeed, \tool is the
first such cryptographic-cost aware compiler.

\paragraph{Scalability (secure code partitioning).} \mpc tools
often do not scale to large functionalities. The reason is that
most \mpc implementations use a circuit-like representation as an
intermediate language. Hence, the largest compute that can be done
securely is upper-bounded by the largest circuit that can fit in the
machine memory\footnote{Using swap and
disk space is feasible but it causes huge slowdown.}. This is a
show-stopper for applications like secure machine learning, secure
prediction, etc. that operate on large amounts of data.
%\aseem{Can we
%cite something here to support this claim?}
\tool addresses the scalability concern using a novel technique that
we call secure code partitioning (or partitioning in short). At
a high level, we decompose the original program into a sequence of small
sub-programs, which are then sequentially processed by \tool, while
appropriately threading the intermediate outputs
along. While this
addresses the scalability concern (i.e., the circuit
sizes of the sub-programs are now small enough to fit in the memory),
we still have to address
the security risk of revealing the intermediate outputs. \tool comes
to the rescue; it automatically inserts the required instrumentation
to ensure security of these intermediate outputs (\sectionref{pipe}). As
we show in our
evaluation, partitioning allows us to program large applications
in \tool. 
%\aseem{What about the differences from Yao pipelining? Worth
%adding here that it is a backend-agnostic technique?}

%% We provide details of partitioning and prove the
%% correctness and security of this transformation
%% in \sectionref{pipe}

\paragraph{Formal guarantees.} We prove formal correctness and
security theorems for our compiler. The correctness theorem relates the
``trusted third party'' semantics of a source
program and the ``protocol'' semantics (the distributed \mpc
semantics that relies on circuit evaluation) of the corresponding
compiled program. The theorem
guarantees that for all well-typed source programs, the two semantics
successfully terminate
(e.g., there are no array index out-of-bounds errors) with identical
observable outputs. For
the security theorem, we formally reduce the security
of our scheme against semi-honest (or ``honest but curious")
adversaries to the semi honest security of the  \mpc back-end. The
theorem provides protection against side-channels arising from
conditionals and memory access patterns.
 We also prove a formal security theorem against semi-honest
 adversaries for our partitioning
scheme (\sectionref{ld} and~\sectionref{pipe}).

\paragraph{Evaluation.} We have implemented \tool
using ABY~\cite{aby} as the
cryptographic back-end.
We compare \tool with Yao-based compilers in~\sectionref{oblivm}
and with specialized protocols in~\sectionref{eval}. 
 We evaluate \tool by implementing a wide range
of secure prediction benchmarks including linear and na\"{i}ve bayes
classifiers, decision trees, deep neural networks, state-of-the-art
classifiers from Tensorflow~\cite{tensorflow}
and \bonsai~\cite{bonsai}, and also the matrix factorization example
from Nikolaenko et al.~\cite{valeriaMatrix}. Our results demonstrate
three key points. First, \tool makes it convenient for general
programmers to write \mpc protocols. E.g. we provide the first \mpc
implementation of \bonsai~\cite{bonsai}, and it was programmed in the
high-level \tool source language by a non-cryptographer. Second, the
performance
of the protocols
generated by \tool are comparable to or better than (up to 19x) their state-of-the-art, hand-crafted implementations. Finally, we
demonstrate the usefulness of partitioning by implementing an application
that requires more than 300 million gates (\sectionref{impl} and~\sectionref{eval}).

\paragraph{Related Work.}
Before ABY~\cite{aby}, several works have proposed
combining secure computation protocols based on homomorphic
encryption and Yao's garbled circuits
(e.g. \cite{barni,blanton,brickell,franz,huang,valeriaMatrix,valeriaRidge,schropferK11}),
and some have also developed tools that allow writing such
combinations (e.g. \cite{bogdanov,lone,tasty,autoS}). However, as Demmler et
al.~\cite{aby} observe, due
 to the high conversion cost between
homomorphic encryption and Yao's garbled circuits, these combined
protocols do not gain much performance over a single
protocol. 
Additionally, these prior works provide informal languages
or libraries that lack formal semantics and static guarantees.
Finally, we focus on the language
features necessary to implement machine learning applications.
In particular, we do not discuss declassification of
intermediate values or indexing into arrays at secret indices.
 Handling them requires additional complexity. For example,
Wysteria~\cite{wysteria} handles the former using dependent types
and ObliVM~\cite{oblivm} uses Oblivious RAM for the latter.
These situations seldomly arise in the machine learning applications that we consider:
intermediate values don't need to be declassified and arrays are traversed in an oblivious manner. We provide a detailed survey of related
work in \sectionref{related}.

%\dg{Cite works like Wysteria~\cite{wysteria} for public inputs and declassification of intermediate values, ObliVM~\cite{oblivm} for hiding secret array indices using Oblivious RAM. But since the focus of this work is machine learning applications, we do not concern ourselves with these language complexities.}
%Can be added to related work
% Kerschbaum et al.~\cite{autoS}
%provide a scheme to automatically assign homomorphic encryption or
%garbled circuits to each operators in a computation that has been expressed as a sequence of dyadic operations.
%They conjecture that the problem is NP hard and gave a linear programming based solution
%as well as a quadratic time greedy heuristic. \tool  uses a simpler constant time heuristic
%to assign arithmetic secret sharing or garbled circuits to sub-computations
%in a richer language with control flow statements such as loops and branches.
 

%% The rest of the paper is structured as follows. We formalize \tool and
%% prove its correctness and security theorems in
%% Section~\ref{sec:ld}. Section~\ref{sec:pipe} details our partitioning
%% scheme and proves its security. Section~\ref{sec:impl}
%% presents the implementation details, and
%% Section~\ref{sec:eval} present a comprehensive evaluation of \tool. We
%% begin with an overview of \tool through an example in
%% Section~\ref{sec:ex}. %\nc{provided a pointer to sec 2 as well.}
%\aseem{Sure :), I did not add it because Sec. 2 was just next line, so
%  I thought it did not need a pointer.}



%% Most works cite garbled
%% circuit based two-party computation as the most practical approach so
%% far and use it as the baseline for comparison of performance. But, it
%% has been shown multiple times that the garbled circuits based approach
%% does not scale for realistic examples. Even for small benchmarks
%% (e.g., Na\"{i}ve Bayes) the size of garbled circuit can be
%% huge. Hence, comparisons are often done on artificial data sizes and
%% garbled circuits are shown to be 500x slower than hand-crafted
%% protocols \cite{shafindss}. This is cited as the motivation for need
%% for specialized protocols for different ML algorithms in secure
%% prediction such as linear regression \cite{shafindss}, decision
%% trees \cite{wu}, deep neural networks \cite{minionn} and matrix
%% factorization \cite{valeriaMatrix}. Our work is based on generic \mpc
%% and we show competitive performance with the specialized protocols
%% contrary to the popular belief. Hence, we believe \tool based
%% implementations can serve as the performance baseline for future works
%% on specialized protocols.


%% an intermdiate language (that is the output of our
%% compiler), and a circuit language for arithmetic and boolean
%% circuits.


%% We give a formal {\em type
%% system} for our language and prove useful properties. We prove that if
%% a program type checks in our language then it would run correctly,
%% e.g., there would not be any array index out of bounds error. The
%% programmer writes a single code describing the inputs from each party
%% and the function to be computed. Our compiler automatically generates
%% separate executables for the server and the client that can be
%% executed in the distributed setting. We prove the {\em correctness}
%% for this compilation, i.e., the outputs obtained in the distributed
%% execution are identical to the stand-alone execution by a trusted
%% third party. For details on formal
%% language, correctness and security theorems see \sectionref{ld}.






%% We demonstrate the generality of our framework by writing a wide range
%% of functionalities in \tool such as linear regression, deep neural
%% networks, decision trees and matrix factorization. 
%% Our evaluations show that automatically generated protocols
%% using \tool are competitive with tailor-made protocols
%% in \cite{shafindss,wu,barni,minionn,secureml,valeriaMatrix}
%% (see \sectionref{eval}). These features allow us to evaluate our
%% system state-of-the-art ML prediction algorithms,
%% e.g., \sectionref{eval} describes evaluation on ML models from a
%% recent ICML paper \cite{bonsai} and Tensorflow \cite{tensorflow}. \\


%% \tool is a programmer-friendly
%% framework to
%% express the function to be computed and uses a mix of arithmetic and
%% boolean \mpc back-ends. We illustrate the above example of $w^Tx
%% >b$ in our language in \figureref{ex-sml}. 
%% As is clear from the code, the programmer can remain completely
%% oblivious of the cryptographic back-end and underlying circuit
%% representation.
%% Also,  the programmer is not required to write keywords such as {\em
%% secret}
%% or {\em public} (that are required by languages like
%% Wysteria \cite{wysteria}).
%%  In contrast, \figureref{ex-aby} illustrates
%% the same example using the framework in ABY \cite{aby}. Here, as
%% already mentioned, the programmer needs to create shares and write the
%% low level gates
%% like $\mathtt{MUL, ADD, CONS, GT}$, etc \TODO{make these fonts
%% consistent}. The programmer is also required
%% to use the appropriate share conversion gates such as $\mathtt{A2Y}$ that
%% converts arithmetic shares to Yao-based boolean shares to be used in
%% comparison after the inner-product computation.
%% Moreover, it is easy to
%% maintain code in \tool and a small change to the functionality requires
%% only a small change in the code. In contrast, working with ABY, even
%% for a small change a developer might have to re-work the whole circuit
%% for efficiency reasons. We elaborate on these examples  in
%% \sectionref{ex}. \\

%% {\aseem{Not sure about this baseline paragraph. Did not touch it.}} 

%Finally, our framework \tool can serve as a meaningful baseline for future works on hand-crafted \mpc protocols for specific functionalities such as \textsc{MiniONN} \cite{minionn}. 
%Prior to our work, garbled circuit based two-party computation is cited as the most practical approach and used as the baseline for comparison of performance. But, it has been observed multiple times this approach does not even scale on realistic examples.

%% We have implemented \tool using ABY~\cite{aby} as the crypto backend.
%% We evaluate our system on wide range of benchmarks for secure
%% prediction \cite{shafindss,wu,barni,secureml,minionn,bonsai} using
%% linear classifier, decision trees and deep neural networks, matrix
%% factorization \cite{valeriaMatrix} and training of deep neural
%% network \cite{secureml}.



%% for secure
%% computation. We provide the first compiler where the developer writes
%% a high-level program and the compiler automatically generates an
%% efficient \mpc protocol that uses a
%% mix of arithmetic and boolean circuits. 
%% It is compatible with any implementation of \mpc that allows a mix of
%% arithmetic and boolean circuits.
%% We focus on semi-honest secure two-party computation and use
%% ABY \cite{aby} as the suitable cryptographic back-end.
%% We provide a formal type system and operational semantics for our
%% language and prove correctness and security of our compiler.
%% Our evaluations show that protocols generated
%% by our compiler match
%% the performance or outperform hand-crafted protocols in most cases (see
%% \sectionref{eval}). \TODO{Add some text to increase enthusiasm about
%% results.} Moreover, unlike previous systems such
%% as \cite{wysteria,aby,cbmcgc}, our system can scale to arbitrarily
%% large computations using our technique of {\em secure code
%% partitioning} (see below). \divya{I think we can skip Yao pipelining
%% here, right?}
%% Hence, we show that our framework achieves all four properties of
%% programmability, generality, performance, and scalability (see below
%% for details).

%% Prior to our work, there have been systems that provide only a subset of the above properties. For instance, using the framework of CBMC-GC \cite{cbmcgc} allows  functionalities to be expressed in high-level language C but it gives poor performance due to its usage of boolean circuits as underlying representation. Similarly, ABY \cite{aby} framework can potentially give good performance (due to mix of boolean and arithmetic circuits) but is not programmable. Yao's garbled circuits \cite{yao} work with boolean circuits and are the most widely used 2PC protocols but they  do not scale to large functions due to large memory requirement.
%% Huang et al. \cite{yao-pipe} gave the technique of pipelining Yao's garbled circuits that enabled scaling to large functions but still gives poor performance due to its use of boolean circuit representation. 
%% To summarize, none of the prior frameworks have been used to evaluate the benchmarks such as secure prediction using linear regression, decision trees, deep neural networks, etc. In fact, it was widely believed that generic 2PC would not scale to the secure prediction and hence, various protocols were hand-crafted for this task \cite{shafindss,wu,secureml,minionn}. Our framework contradicts this belief (see below).

%% In this work, we solve the design problem to provide a user friendly language that is expressive and can be compiled to efficient secure 2PC protocols.
%% Programmability in \tool and performance of generated protocols has enabled us to write large programs whose circuits  exhaust the memory of the system (~32 GB). To combat this issue, we give a technique called secure code partitioning (see below) that allows scaling of \tool to arbitrary size computations.

%% Below, we describe the salient features of \tool in more detail. \\




%% \aseem{Checkpointing here. Three more paragraphs to come, one for
%% summary of implementation and evaluation, one for related work, and
%% one for baseline argument.}

%\begin{figure}
\begin{verbatim}
uint temp = 1<<31-1;
uint j = 0;

for uint i = [0:30] {
  j = j + (w[i] * x[i]);
}
output((j > temp) ? 1 : 0);

\end{verbatim}
\caption{Code in \tool}
\end{figure}
%\begin{figure}
\begin{lstlisting}[language=C, mathescape=true]
//circuit builders for arithmetic and boolean
$\mathtt{
Circuit\;*ycirc = s[S\_YAO]\rightarrow GetCircuitBuildRoutine();
}$
$\mathtt{
Circuit\;*acirc = s[S\_ARITH]\rightarrow GetCircuitBuildRoutine();
}$
$\dots$
$\mathtt{
if(role == SERVER)\;\{ 
}$
  //Put gates to read w and b
$\mathtt{
\}\;else\;\{\;\;\text{{\color{red}//role == CLIENT}}
}$  
  //Put gates to read x
$\mathtt{
\}
}$  

$\mathtt{
for(uint32\_t\;i\;=\;0;\;i\;<\;30;\;i = i + 1)\;\{\;\;\text{\color{red}//acc = w}^{\text{{\color{red}T}}}\text{\color{red}x}~\label{line:dotproductloop}
}$
$\mathtt{
\;\;\;\;share\;*a\_t\_0 = acirc\rightarrow PutMULGate(a\_w[i], a\_x[i]);~\label{line:dotmulgate}
}$
$\mathtt{
\;\;\;\;a\_acc = acirc\rightarrow PutADDGate(a\_acc, a\_t\_0 );~\label{line:dotaddgate}
}$
$\mathtt{
\}
}$

//convert acc and b from arithmetic to boolean
$\mathtt{
share\;*y\_acc = ycirc\rightarrow PutA2YGate(a\_acc);~\label{line:convacc}
}$
$\mathtt{
share\;*y\_b = ycirc\rightarrow PutA2YGate(a\_b);~\label{line:convb}
}$

$\mathtt{
share\;*y\_pred = ycirc\rightarrow PutGTGate(y\_acc, y\_b);~\label{line:condyaobegin}
}$
$\mathtt{
uint32\_t\;one = 1;
}$
$\mathtt{
share\;*y\_1 = ycirc\rightarrow PutCONSGate(one, bitlen);
}$
$\mathtt{
uint32\_t\;zero = 0 ;
}$
$\mathtt{
share\;*y\_0 = ycirc\rightarrow PutCONSGate(zero, bitlen);
}$
$\mathtt{
share\;*y\_t = ycirc\rightarrow PutMUXGate(y\_pred, y\_1, y\_0);~\label{line:condyaoend}
}$

$\mathtt{
share\;*y\_out = ycirc\rightarrow PutOUTGate(y\_t, CLIENT);
}$
$\mathtt{
party\rightarrow ExecCircuit();
}$

$\mathtt{
if(role == CLIENT) \{\;\;\text{\color{red}//only to the client}
}$
$\mathtt{
\;\;\;\;uint32\_t \_o = y\_out\rightarrow get\_clear\_value\langle uint32\_t \rangle();
}$
$\mathtt{
\}
}$
\end{lstlisting}
\caption{\tool compiler (partial) output for Figure~\ref{fig:ex-sml}}
\label{fig:ex-aby}
\end{figure}


\section{\tool Overview}
\label{sec:ex}
%$w^Tx>0$, mnist logistic regression
%% \divya{add reference to before and later... as mentioned we would run evaluations on secure prediction of different models. Here we describe scenario in detail..}

\begin{figure}
  %\vspace{-20pt}
  %\begin{center}
  \includegraphics[width=0.45\textwidth]{toolchain}
  %\end{center}
  %\vspace{-20pt}
\caption{\tool toolchain}
\label{fig:toolchain}
\end{figure}

Figure~\ref{fig:toolchain} shows an overview of the \tool
toolchain. We give a brief overview of each of these phases below.

\subsubsection*{Source language}
Consider the example $w^Tx >b$ from Section~\ref{sec:intro}, where
$w$ and $b$ constitute the server's input (a classifier) and $x$ is
the client's input vector. Figure~\ref{fig:ex-sml} shows \tool code
for this example. The code first reads the inputs of the two parties
using the \ls{input1} and \ls{input2} expressions. It then uses a
\ls{for} loop to compute \ls{acc}, the dot product of \ls{w} and
\ls{x}. Finally, the code outputs the result of comparing \ls{acc}
with \ls{b} only to the client.

\tool source language is a simple, imperative language that enables
the programmers to express \mpc computations in terms of their
``ideal'' functionalities, without dealing with any cryptographic
details. The languages provides multi-dimensional arrays, conditional
expressions (the ternary $?\::$ operator), \ls{for}~loops,
\ls{if}~statements, and special syntax for input/output.

\begin{figure}
\begin{lstlisting}[language=C, mathescape=true]
//circuit builders for arithmetic and boolean
$\mathtt{
Circuit\;*ycirc = s[S\_YAO]\rightarrow GetCircuitBuildRoutine();
}$
$\mathtt{
Circuit\;*acirc = s[S\_ARITH]\rightarrow GetCircuitBuildRoutine();
}$
$\dots$
$\mathtt{
if(role == SERVER)\;\{ 
}$
  //Put gates to read w and b
$\mathtt{
\}\;else\;\{\;\;\text{{\color{red}//role == CLIENT}}
}$  
  //Put gates to read x
$\mathtt{
\}
}$  

$\mathtt{
for(uint32\_t\;i\;=\;0;\;i\;<\;30;\;i = i + 1)\;\{\;\;\text{\color{red}//acc = w}^{\text{{\color{red}T}}}\text{\color{red}x}~\label{line:dotproductloop}
}$
$\mathtt{
\;\;\;\;share\;*a\_t\_0 = acirc\rightarrow PutMULGate(a\_w[i], a\_x[i]);~\label{line:dotmulgate}
}$
$\mathtt{
\;\;\;\;a\_acc = acirc\rightarrow PutADDGate(a\_acc, a\_t\_0 );~\label{line:dotaddgate}
}$
$\mathtt{
\}
}$

//convert acc and b from arithmetic to boolean
$\mathtt{
share\;*y\_acc = ycirc\rightarrow PutA2YGate(a\_acc);~\label{line:convacc}
}$
$\mathtt{
share\;*y\_b = ycirc\rightarrow PutA2YGate(a\_b);~\label{line:convb}
}$

$\mathtt{
share\;*y\_pred = ycirc\rightarrow PutGTGate(y\_acc, y\_b);~\label{line:condyaobegin}
}$
$\mathtt{
uint32\_t\;one = 1;
}$
$\mathtt{
share\;*y\_1 = ycirc\rightarrow PutCONSGate(one, bitlen);
}$
$\mathtt{
uint32\_t\;zero = 0 ;
}$
$\mathtt{
share\;*y\_0 = ycirc\rightarrow PutCONSGate(zero, bitlen);
}$
$\mathtt{
share\;*y\_t = ycirc\rightarrow PutMUXGate(y\_pred, y\_1, y\_0);~\label{line:condyaoend}
}$

$\mathtt{
share\;*y\_out = ycirc\rightarrow PutOUTGate(y\_t, CLIENT);
}$
$\mathtt{
party\rightarrow ExecCircuit();
}$

$\mathtt{
if(role == CLIENT) \{\;\;\text{\color{red}//only to the client}
}$
$\mathtt{
\;\;\;\;uint32\_t \_o = y\_out\rightarrow get\_clear\_value\langle uint32\_t \rangle();
}$
$\mathtt{
\}
}$
\end{lstlisting}
\caption{\tool compiler (partial) output for Figure~\ref{fig:ex-sml}}
\label{fig:ex-aby}
\end{figure}


\subsubsection*{\tool compiler}
\tool compiler takes as input a source program and produces a C++
program as output. Figure~\ref{fig:ex-aby} shows the output code for
the example in Figure~\ref{fig:ex-sml}. The output program contains
contains party-specific code for inputs and outputs
(\ls{role == SERVER} and \ls{role == CLIENT}), and common code for the
computation.

The compiler splits the input
program into \emph{public} and \emph{secret} components. The public
components translate into regular C++ code, while the secret
components translate into API calls into our crypto back-end
(ABY). For instance, for the code in Figure~\ref{fig:ex-sml}, the \tool
compiler realizes that the array index \ls{i} in the dot product loop
is public, and hence the array accesses need not be compiled
obliviously. Therefore, it compiles the \ls{for}~loop into a C++
\ls{for}~loop that will be executed in-clear
(line~\ref{line:dotproductloop}).

Within the secret components, the \tool compiler is ``cryptographic
cost-aware'', and appropriately picks either arithmetic or boolean
circuit representations for different sub-componenets. For example,
the compiler realizes that the dot product computation is more
efficient in the arithmetic representation, and therefore it builds
the corresponding circuit using the arithmetic circuit builder
\ls{acirc} (lines~\ref{line:dotmulgate} and~\ref{line:dotaddgate}). On
the other hand, since the comparison with \ls{b}, and the conditional
expression computation are more efficient in the boolean
representation, the \tool compiler uses the Yao circuit builder
\ls{ycirc} to build the corresponding circuits
(lines~\ref{line:condyaobegin} to~\ref{line:condyaoend})\footnote{As
  we mention in
  Section~\ref{sec:impl}, in our usage of ABY, boolean is synonymous
  with Yao.}.

Importantly, conversions between the arithmetic and boolean
represenation require share-conversions. The \tool compiler also
instruments these conversion gates accordingly. For example, in
lines~\ref{line:convacc} and~\ref{line:convb}, the compiler converts
\ls{a_acc} and \ls{a_b} to boolean representation, before they are
input to the comparison and multiplexer circuits.

\subsubsection*{Circuit generation and evaluation} The next step in
\tool is to
compile the output C++ code and execute it. Doing so reduces away the
public parts of the program, including the array accesses, and
generates a \mpc circuit comprising of arithmetic and boolean gates,
with appropriate conversion gates. The circuit is then evaluated using
a \mpc protocol.

%% Consider a  cloud service provider Bob that wants to provide a service to diagnose
%% whether a patient has breast cancer or not. Bob trains a machine learning classification model
%% that results in a vector $w$ (say of length 30) and a scalar $b$.
%% This model is the intellectual property of Bob and he  wants to keep it secret.
%% Given a patient's medical report, in the form of a vector $x$ (also of length 30),
%% the classifer predicts that the patient has breast cancer if $w^Tx>b$.
%% However, for this task Bob needs access to $x$, which is private data of a customer.

%% A potential customer Alice might not want to reveal $x$ to Bob because of privacy concerns.
%% And Bob does not want to reveal $w$ and $b$ because then Alice can steal the model. \divya{Use HIPAA compliance; as it can leak information of patients that was used in training; more compelling argument. See Shafi intro.}
%% SMC can help Alice and Bob compute $w^Tx>b$ securely, such that Alice receives the classifier's
%% prediction and Bob learns nothing about $x$. Moreover, Alice does not learn anything more about $w$
%% and $b$ than what is revealed by the prediction. 
%% \nc{This is a bit repetitive from the intro, isnt it? Should we have another example perhaps?}
%% To implement this system, Bob can write the code in \tool and this code is shown in Figure~\ref{fig:ex-sml}.
%% The expression {\tt input1} reads a value from Bob and {\tt input2} reads from Alice.
%% The language has arrays and simple loops. A loop {\tt for i = [0:N]} repeats its
%% body {\tt N} times, the loop counter {\tt i} is assigned {\tt 0} in the first iteration,
%% and is incremented by one after each iteration. Unbounded {\tt while} loops are problematic
%% from a cryptographic standpoint and our language only permits these simple {\tt for} loops. 
%% The first loop in Figure~\ref{fig:ex-sml} reads the model $w$ from Bob, the second loop reads
%% Alice's medical report $x$, and the third loop computes the dot product $w^Tx$.
%% The last {\tt output} statement sends the result of the comparison $w^Tx>b$ to Alice.
%% In particular, the ternary ``{\tt ? :}" operator performs a branch and the result is the second (third)
%% argument if the first argument is true (false).

%% The \tool compiler is fully automatic and compiles the code described in Figure~\ref{fig:ex-sml} to the C++ code that makes calls to the ABY library  in
%% Figure~\ref{fig:ex-aby}. An alternative to using \tool is to directly write this C++ code.
%% However, this code is much more complex than the implementation written in \tool.
%% Therefore, it is difficult for Alice and bob to verify its correctness and security.
%% Unlike \tool implementations, the ``secret" variables containing private data ({\tt w,x,b,acc}) need to be handled differently compared to ``public" variables such as loop counters ({\tt i}). In particular, the former are manipulated by the ABY library and the latter are manipulated like standard  C++ variables.
%% Additionally, the C++ code branches on  the special variable
%% {\tt role}  to decide whether the enclosing code is executed by the {\tt SERVER} Bob or by
%% the {\tt CLIENT} Alice. In Figure~\ref{fig:ex-sml}, the developer does not need to
%% manipulate {\tt role} explicitly. In addition to simplifying the implementation, \tool prevents developers from writing buggy code such as\\
%% \verb+if(role == SERVER) {...manipulate x...}+

%% Furthermore, the code in Figure~\ref{fig:ex-aby} is too low level.
%% In particular, it is first builds a circuit that represents the computation
%% $w^Tx>b$ and then executes it via a call to {\tt ExecCircuit}.
%% To build a circuit, the code adds multiplication gates ({\tt PutMULGate})
%% and addition gates ({\tt PutADDGate}) for the dot product.
%% Subsequently, it uses a ``greater than" gate ({\tt PutGTGate})
%% and a multiplexer gate ({\tt PutMUXGate}) for the comparison with $b$.
%% For efficiency, the operations for dot product need to be performed using 
%% arithmetic circuits. However, ABY's arithmetic circuits cannot express
%% comparisons and these need  boolean circuits.
%% A program that uses both arithmetic circuit and boolean circuits
%% requires conversion gates that help  interconvert between arithmetic
%% representations and boolean representations ({\tt PutA2YGate}).
%% Bigger computations often require multiple conversions and the
%% developer effort can quickly become significant.

%% Moreover, it is difficult to maintain the code in Figure~\ref{fig:ex-aby}.
%% For example, if the multiplication is changed to a bitwise-or then,
%% in the absence of \tool, an efficiency-conscious developer would need to  use boolean circuits everywhere and would be tasked with removing all arithmetic circuits and the interconversion gates (the cost of  conversion overweighs the gains of performing an arithmetic addition instead of a boolean addition). 
%%  With \tool, the developer needs to change only a single character (replace {\tt *} in Figure~\ref{fig:ex-sml} with {\tt |}) to achieve the same effect. 
%% On this example and other small examples, we have found that the compiler generated C++ code from a \tool implementation is as efficient as manually written C++ code. In particular, for $w^Tx>b$, the compiler automatically generates code that uses arithmetic circuits for dot product and inserts appropriate conversion gates. Finally, although this paper uses ABY for evaluation, the  compiler can be easily retargeted to generate code for other cryptography backends.


%% The \tool compiler provides strong static guarantees. For example, a well typed
%% \tool program is guaranteed to execute to termination without errors. A \tool program cannot go into non-termination or dereference illegal memory, i.e., no buffer overflows or underflows can happen at run time. Production compilers such as {\tt gcc} provide no such guarantees
%% for the code in Figure~\ref{fig:ex-aby}.
%% The primary reason that we are able to achieve these guarantees is because
%% \tool has been designed to be verifiable and is backed by formal semantics. 
%% Furthermore, the compiler output is guaranteed to be a cryptographically secure
%% implementation of the functionality declared by the \tool code. 
%% Moreover, the \tool compiler also generates a C implementation that can be run natively on a single machine for functional testing. 

%% To summarize, \tool raises the level of abstraction, provides strong static guarantees, and generates efficient code automatically.

\newcommand{\kw}[1]{{\lstinline[basicstyle=\small\color{blue}]{#1}}}
\newcommand{\ftext}[1]{\text{\small{#1}}}
\newcommand{\cond}[3]{\ensuremath{{{#1}\:?\:{#2}\::{#3}}}}
\newcommand{\for}[4]{\ensuremath{\kw{for}\:{#1}\:\kw{in}\:[{#2}, {#3}]\:\kw{do}\:{#4}}}
\newcommand{\ite}[3]{\ensuremath{\kw{if}({#1}, {#2}, {#3})}}
\newcommand{\loops}[3]{\ensuremath{\kw{while}\:{#1} \leq {#2}\:\kw{do}\:{#3}}}

\section{Formal development}
\label{sec:ld}

In this section we prove the correctness and security of our compiler.
%
We first formalize the source and target languages. Our source runtime
semantics is a model of the ideal, trusted third-party semantics, and
generates observations corresponding to the values revealed to the
parties.
%
The target language semantics (a model of the C++ code generated by
our compiler implementation) ``computes away'' the public parts of the
compiled program, generating a secure computation circuit.
%
Finally, we formalize the circuit semantics that computes the
generated circuit, and like the source semantics, emits observations.

We then present the compilation judgments. To prove the correctness
of our compiler, we prove that it preserves the observations. For
security of our compiler, we reduce the security argument to the
security of the cryptographic protocol used to compute the secure
computation circuit.

We present only selected parts of our formalization for space
reasons. Full definitions and proofs can be found in the supplementary
material submitted along with the paper.


\begin{figure}
  \small
  \[
  \begin{array}{rrcl}
    \ftext{Base type} & \sigma &::=& \kw{uint} \mid \kw{bool}\\
    \ftext{Type} & \psi &::=& \sigma \mid \sigma[n]\\
    \ftext{Constant} & c &::=& n \mid \top \mid \bot\\
    \ftext{Expression} & e &::=& c \mid x \mid e_{1} + e_{2} \mid e_{1} > e_{2} \mid \cond{e_{1}}{e_{2}}{e_{3}}\\
    & &\mid& [\overline{e_{i}}]_{n} \mid x[e] \mid \kw{in}_{j}\\
    \ftext{Statement} & s &::=& \psi\:x = e \mid x := e \mid \for{x}{n_{1}}{n_{2}}{s}\\
    & & \mid& x[e_{1}] := e_{2} \mid \ite{e}{s_{1}}{s_{2}} \mid \kw{out}\:e \mid s_{1}; s_{2}\\
    & & \mid& \loops{x}{n}{s}
  \end{array}
  \]
\caption{Source language}
\label{fig:srclang}
\end{figure}

\begin{figure}
  \small
  \fbox{$\rho \vdash e \Downarrow v$}
  \[
  \\
  \begin{array}{c}
    \inferrule*[lab={\footnotesize{E-Var}}]
               {
               }
               {
                 \rho \vdash x \Downarrow \rho(x)
               }
               
               \hspace{0.1cm}
               
    \inferrule*[lab={\footnotesize{E-Add}}]
               {
                 \forall i \in \{1, 2\}.\:\rho \vdash e_{i} \Downarrow n_{i}
               }
               {
                 \rho \vdash e_{1} + e_{2} \Downarrow n_{1} + n_{2}
               }

               \hspace{0.1cm}
               
    \inferrule*[lab={\footnotesize{E-Read}}]
               {
                 \rho \vdash x \Downarrow [\overline{c_{i}}]_{n_{1}} \\\\
                 \rho \vdash e \Downarrow n \quad n < n_{1}
               }
               {
                 \rho \vdash x[e] \Downarrow c_{n}
               }
\\\\
    \inferrule*[lab={\footnotesize{E-Arr}}]
               {
                 \forall i \in \{0 \dots n - 1\}.\:\rho \vdash e_{i} \Downarrow c_{i}
               }
               {
                 \rho \vdash [\overline{e_{i}}]_{n} \Downarrow [\overline{c_{i}}]_{n}
               }
               \quad
    \inferrule*[lab={\footnotesize{E-Inp}}]
               {
               }
               {
                 \rho \vdash \kw{in}_{j} \Downarrow c
               }
  \end{array}
  \]
  \\\\
    \fbox{$\rho \vdash c \Downarrow \rho'; O$}
  \[
  \\
  \begin{array}{c}
    \inferrule*[lab={\footnotesize{E-Decl}}]
               {
                 \rho \vdash e \Downarrow v
               }
               {
                 \rho \vdash \psi\:x = e \Downarrow \rho, x \mapsto v; \cdot
               }
               
               \hspace{0.1cm}

    \inferrule*[lab={\footnotesize{E-LoopT}}]
               {
                 \rho(x) > n
               }
               {
                 \rho \vdash \loops{x}{n}{s} \Downarrow \rho; \cdot
               }

               \\\\
               
    \inferrule*[lab={\footnotesize{E-LoopI}}]
               {
                 \rho(x) \leq n\\\\
                 \rho \vdash s \Downarrow \rho_{1}; O_{1}\\\\
                 \rho_{2} = [\rho_{1}]_{\mathsf{dom}(\rho)}[x \mapsto \rho_{1}(x) + 1]\\\\
                 \rho_{2} \vdash \loops{x}{n}{s} \Downarrow \rho'; O_{2}
               }
               {
                 \rho \vdash \loops{x}{n}{s} \Downarrow \rho'; O_{1}, O_{2}
               }

               \hspace{0.2cm}

    \inferrule*[lab={\footnotesize{E-If}}]
               {
                 \rho \vdash e \Downarrow c\\\\
                 c = \top \Rightarrow s = s_{1}\\\\
                 c = \bot \Rightarrow s = s_{2}\\\\
                 \rho \vdash s \Downarrow \rho'; O
               }
               {
                 \rho \vdash \ite{e}{s_{1}}{s_{2}} \Downarrow \rho'; O
               }

               \\\\
               
    \inferrule*[lab={\footnotesize{E-For}}]
               {
                 \rho, x \mapsto n_{1} \vdash \loops{x}{n_{2}}{s} \Downarrow \rho_{1}; O
               }
               {
                 \rho \vdash \for{x}{n_{1}}{n_{2}}{s} \Downarrow \rho_{1} - \{x\}; O
               }


               \hspace{0.2cm}

    \inferrule*[lab={\footnotesize{E-Out}}]
               {
                 \rho \vdash e \Downarrow c
               }
               {
                 \rho \vdash \kw{out}\:e \Downarrow \rho; c
               }

\end{array}
  \]
\caption{Source semantics (selected rules)}
\label{fig:srcsem}
\end{figure}

\subsubsection*{Source language}Our source language
(Figure~\ref{fig:srclang}) is a simple imperative language. Types
$\psi$ in the language consist of the base types $\sigma$, and arrays
of base types $\sigma[n]$, where $n$ is the array length. Though we
model only one dimensional arrays, our implementation supports higher
dimensional arrays as well. Expressions in the language include the
integer constants $n$, \kw{bool} constants $\top$ and $\bot$,
variables $x$, binary operations $e_{1} + e_{2}$ and $e_{1} > e_{2}$,
conditionals $\cond{e}{e_{1}}{e_{2}}$, array literals 
$[\overline{e_{i}}]_{n}$\footnote{We write $\overline{e}$ (and
  similarly for other symbols) to denote a sequence of expressions.
The length of the sequence is usually clear from the context.}, and
array reads $x[e]$. The expression $\kw{in}_{j}$ denotes input from
party $j$. The statements $s$ in the language comprise of variable
declarations and assignments ($\psi\:x = e$ and $x := e$ resp.),
\kw{for} loops, array writes ($x[e_{1}] := e_{2}$), \kw{if}
statements, and sequence of statements ($s_{1}; s_{2}$). The statement
$\kw{out}\:e$ denotes revealing the value of $e$ to the
parties. The \kw{while} statement is an internal syntax that is not
exposed to the programmer.

The runtime semantics for the source language is shown in
Figure~\ref{fig:srcsem}. Values $v$, runtime environments $\rho$, and
observations $O$ are defined as follows:

\vspace{0.2cm}
$
\small
\begin{array}{rrcl}
    \ftext{Value} & v &::=& c \mid [\overline{c_{i}}]_{n}\\
    \ftext{Runtime environment} & \rho &::=& \cdot \mid \rho[x \mapsto v]\\
    \ftext{Observation} & O & ::= & \cdot \mid c, O \\
\end{array}
$

\vspace{0.2cm}
Values consist of constants and array values. Runtime environment
$\rho$ maps variables to values. Observations are sequences of
constants.

The judgment $\rho \vdash e \Downarrow v$ denotes the big-step
evaluation of an expression~$e$ to a value~$v$ under the runtime
environment~$\rho$. Rule ({\sc{E-Var}}) looks up the value of $x$ in
the environment. Rule ({\sc{E-Add}}) inductively evaluates $e_{1}$ and
$e_{2}$ and evaluates to their addition. Rule ({\sc{E-Read}})
evaluates an array read operation. It first evaluates $x$ to an array
value $[\overline{c_{i}}]_{n_{1}}$, and $e$ to an \kw{uint} value
$n$. It then returns $c_{n}$, the $n$-th index value in the array,
provided $n < n_{1}$, the length of the array. Rule ({\sc{E-Inp}})
evaluates to some constant $c$ denoting party $j$'s input. We model
the inputs to be base constants, an array input can be written in the
language as $[\kw{in}_{j}]_{n}$, which can then evaluate using the
rule ({\sc{E-Arr}}). The remaining rules are straightforward, and are
elided for space reasons.

The judgment $\rho \vdash s \Downarrow \rho'; O$ represents the
big-step evaluation of an statement $s$ under environment $\rho$
producing a new environment $\rho'$ and observations $O$. Rule
({\sc{E-Decl}}) evaluates the expression $e$ to $v$, and returns the
updated environment $\rho[x \mapsto v]$, with empty observations. Rule
({\sc{E-If}}) evaluates the guard expression, and then evaluates
either $s_{1}$ or $s_{2}$ accordingly. $\kw{for}$ statements evaluate
through the internal $\kw{while}$ syntax. Specifically, rule
({\sc{E-For}}) updates $\rho$ with the initial counter value $n_{1}$,
evaluates $\loops{x}{n_{2}}{s}$ to $\rho_{1}; O$, and returns
$\rho_{1} - \{x\}$ (removing $x$ from the environment) and $O$. Rule
({\sc{E-LoopI}}) shows the inductive case for \kw{while}
statements, when $\rho(x) \leq n$. The rule evaluates $s$, producing
$\rho_{1}; O_{1}$. It then restricts $\rho_{1}$ to the domain of
$\rho$ ($[\rho_{1}]_{\mathsf{dom}(\rho)}$) to remove the variables
added by $s$, increments the value if $x$, and evaluates the
\kw{while} statement under this updated environment. Rule
({\sc{E-LoopT}}) is the termination case for \kw{while} statements,
when $\rho(x) > n$. Finally, the rule ({\sc{E-Out}}) evaluates the
expression, and adds it value to the observations.

\subsubsection{Target language}

\begin{theorem}[Correctness of the compiler]
  
\end{theorem}


\newcommand{\lcond}[4]{\ensuremath{{{#2}\:?_{{#1}}\:{#3}\::{#4}}}}
%\newcommand{\for}[4]{\ensuremath{\kw{for}\:{#1}\:\kw{in}\:[{#2}, {#3}]\:\kw{do}\:{#4}}}
%\newcommand{\ite}[3]{\ensuremath{\kw{if}({#1}, {#2}, {#3})}}
%\newcommand{\loops}[3]{\ensuremath{\kw{while}\:{#1} \leq {#2}\:\kw{do}\:{#3}}}

\begin{figure}
  \small
  \[
  \begin{array}{rrcl}
    \ftext{Secret label} & m &::=& \mathcal{A} \mid \mathcal{B}\\
    \ftext{Label} & \ell &::=& \mathcal{P} \mid m\\
    \ftext{Type} & \tau &::=& \sigma^{\ell} \mid \sigma^{\ell}[n]\\
    \ftext{Expression} & \widetilde{e} &::=& c \mid x \mid \widetilde{e_{1}} +_{\ell} \widetilde{e_{2}} \mid \widetilde{e_{1}} >_{\ell} \widetilde{e_{2}} \\
    & & \mid & \lcond{\ell}{{\widetilde{e}}}{{\widetilde{e_{1}}}}{{\widetilde{e_{2}}}} \mid x[\widetilde{e}] \mid [\overline{\widetilde{e_{i}}}]_{n} \mid \kw{in}^{m}_{j} \mid \widetilde{e} \rhd m\\
    \ftext{Statement} & \widetilde{s} &::=& \tau\:x = \widetilde{e} \mid x := \widetilde{e} \mid \dots \mid \widetilde{s_{1}}; \widetilde{s_{2}} \mid \dots\\
  \end{array}
  \]
\label{fig:tgtlang}
\caption{Target language}
\end{figure}

\begin{figure}
  \small
  \[
  \begin{array}{rrcl}
    \ftext{Wire id} & r &&\\
    \ftext{Circuit gate} & \kappa & ::= & r \mid \kw{in}^{m}_{j} \mid \kw{add}\:\kappa_{1}\:\kappa_{2} \mid \kw{gt}\:\kappa_{1}\:\kappa_{2}\\
    & & \mid & \kw{mux}\:\kappa\:\kappa_{1}\:\kappa_{2} \mid \widetilde{w} \rhd m\\
    \ftext{Base value} & \widetilde{w} & ::= & c \mid \kappa\\
    \ftext{Value} & \widetilde{v} & ::= & \widetilde{w} \mid [\overline{\widetilde{w}_{i}}]_{n}\\
    \ftext{Circuit} & \chi & ::= & \cdot \mid \kw{bind}\:\kappa\:r \mid \kw{out}\:\kappa \mid \kappa_{1}; \kappa_{2}
  \end{array}
  \]
\label{fig:tgtruntime}
\caption{Target runtime}
\end{figure}


\begin{figure}
  \small
  \fbox{$\widetilde{\rho} \vdash \widetilde{e} \Downarrow \widetilde{v}$}
  \[
  \\
  \begin{array}{c}
    \inferrule*[lab={\footnotesize{S-Var}}]
               {
               }
               {
                 \widetilde{\rho} \vdash x \Downarrow \widetilde{\rho}(x)
               }
               
    \inferrule*[lab={\footnotesize{S-PAdd}}]
               {
                 \forall i \in \{1, 2\}.\:\widetilde{\rho} \vdash \widetilde{e_{i}} \Downarrow n_{i}
               }
               {
                 \widetilde{\rho} \vdash \widetilde{e_{1}} +_{\mathcal{P}} \widetilde{e_{2}} \Downarrow n_{1} + n_{2}
               }
               
    \inferrule*[lab={\footnotesize{S-Read}}]
               {
                 \widetilde{\rho} \vdash x \Downarrow [\overline{\widetilde{w_{i}}}]_{n_{1}} \\\\
                 \widetilde{\rho} \vdash \widetilde{e} \Downarrow n \quad n < n_{1}
               }
               {
                 \widetilde{\rho} \vdash x[\widetilde{e}] \Downarrow \widetilde{w_{n}}
               }\\\\
    \inferrule*[lab={\footnotesize{S-SAdd}}]
               {
                 \forall i \in \{1, 2\}.\:\widetilde{\rho} \vdash \widetilde{e_{i}} \Downarrow \kappa_{i}
               }
               {
                 \widetilde{\rho} \vdash \widetilde{e_{1}} +_{\mathcal{A}} \widetilde{e_{2}} \Downarrow \kw{add}\:\kappa_{1}\:\kappa_{2}
               }

               \hspace{0.3cm}

    \inferrule*[lab={\footnotesize{S-SGt}}]
               {
                 \forall i \in \{1, 2\}.\:\widetilde{\rho} \vdash \widetilde{e_{i}} \Downarrow \kappa_{i}
               }
               {
                 \widetilde{\rho} \vdash \widetilde{e_{1}} >_{\mathcal{B}} \widetilde{e_{2}} \Downarrow \kw{gt}\:\kappa_{1}\:\kappa_{2}
               }\\\\
               \inferrule*[lab={\footnotesize{S-SCond}}]
               {
                 \forall i \in \{1, 2, 3\}.\:\widetilde{\rho} \vdash \widetilde{e_{i}} \Downarrow \kappa_{i}
               }
               {
                 \widetilde{\rho} \vdash \lcond{\mathcal{B}}{\widetilde{e_{1}}}{\widetilde{e_{2}}}{\widetilde{e_{3}}} \Downarrow \kw{mux}\:\kappa_{1}\:\kappa_{2}\:\kappa_{3}
               }
               \hspace{0.2cm}
    \inferrule*[lab={\footnotesize{S-Coerce}}]
               {
                 \widetilde{\rho} \vdash \widetilde{e} \Downarrow \widetilde{w}
               }
               {
                 \widetilde{\rho} \vdash \widetilde{e} \rhd m \Downarrow \widetilde{w} \rhd m
               }
               \\\\
               %% \inferrule*[lab={\footnotesize{S-Arr}}]
               %% {
               %%   \forall i \in \{0 \dots n - 1\}.\:\widetilde{\rho} \vdash \widetilde{e_{i}} \Downarrow \widetilde{w_{i}}
               %% }
               %% {
               %%   \widetilde{\rho} \vdash [\overline{\widetilde{e_{i}}}]_{n} \Downarrow [\overline{\widetilde{w_{i}}}]_{n}
               %% }

               \inferrule*[lab={\footnotesize{S-PCond}}]
               {
                 \widetilde{\rho} \vdash \widetilde{e} \Downarrow c\\\\
                 c = \top \Rightarrow \widetilde{e'} = \widetilde{e_{1}}\\
                 c = \bot \Rightarrow \widetilde{e'} = \widetilde{e_{2}}\\\\
                 \widetilde{\rho} \vdash \widetilde{e'} \Downarrow \widetilde{v}
               }
               {
                 \widetilde{\rho} \vdash \lcond{\mathcal{P}}{\widetilde{e}}{\widetilde{e_{1}}}{\widetilde{e_{2}}} \Downarrow \widetilde{v}
               }
               %\hspace{0.5cm}
               
    \inferrule*[lab={\footnotesize{S-Inp}}]
               {
               }
               {
                 \widetilde{\rho} \vdash \kw{in}^{m}_{j} \Downarrow \kw{in}^{m}_{j}
               }
  \end{array}
  \]
  \\\\
    \fbox{$\widetilde{\rho} \vdash \widetilde{c} \Downarrow \widetilde{\rho'}; \chi$}
  \[
  \\
  \begin{array}{c}
    \inferrule*[lab={\footnotesize{S-DeclC}}]
               {
                 \widetilde{\rho} \vdash \widetilde{e} \Downarrow \widetilde{v}\\\\
                 \widetilde{v} = c \vee \widetilde{v} = [\overline{c_{i}}]_{n}\\\\
                 \widetilde{\rho'} = \widetilde{\rho}, x \mapsto \widetilde{v}
               }
               {
                 \widetilde{\rho} \vdash \tau\:x = \widetilde{e} \Downarrow \widetilde{\rho'}; \cdot
               }
               
               \hspace{0.5cm}

    \inferrule*[lab={\footnotesize{S-DeclCkt}}]
               {
                 \widetilde{\rho} \vdash \widetilde{e} \Downarrow \kappa\quad
                 \mathsf{fresh}\:r\\\\
                 \widetilde{\rho'} = \widetilde{\rho}, x \mapsto r\quad
                 \chi = \kw{bind}\:\kappa\:r
               }
               {
                 \widetilde{\rho} \vdash \tau\:x = \widetilde{e} \Downarrow \widetilde{\rho'}; \chi
               }

\\\\
    \inferrule*[lab={\footnotesize{S-DeclCktA}}]
               {
                 \widetilde{\rho} \vdash \widetilde{e} \Downarrow [\overline{\kappa_{i}}]_{n}\\\\
                 \forall i \in \{0 \dots n - 1\}.\:\mathsf{fresh}\:r_{i}\\\\
                 \widetilde{\rho'} = \widetilde{\rho}, x \mapsto [r_{i}]_{n}\quad
                 \chi = \overline{\kw{bind}\:\kappa_{i}\:r_{i}}
               }
               {
                 \widetilde{\rho} \vdash \tau\:x = \widetilde{e} \Downarrow \widetilde{\rho'}; \chi
               }

               \hspace{0.2cm}
               
    \inferrule*[lab={\footnotesize{S-Out}}]
               {
                 \widetilde{\rho} \vdash \widetilde{e} \Downarrow \kappa
               }
               {
                 \widetilde{\rho} \vdash \kw{out}\:\widetilde{e} \Downarrow \widetilde{\rho}; \kw{out}\:\kappa
               }

               \\\\
    \inferrule*[lab={\footnotesize{S-If}}]
               {
                 \widetilde{\rho} \vdash \widetilde{e} \Downarrow c\\\\
                 c = \top \Rightarrow \widetilde{s} = \widetilde{s_{1}}\\\\
                 c = \bot \Rightarrow \widetilde{s} = \widetilde{s_{2}}\\\\
                 \widetilde{\rho} \vdash \widetilde{s} \Downarrow \widetilde{\rho'}; \chi
               }
               {
                 \widetilde{\rho} \vdash \ite{\widetilde{e}}{\widetilde{s_{1}}}{\widetilde{s_{2}}} \Downarrow \widetilde{\rho'}; \chi
               }

               \hspace{0.5cm}
               
    \inferrule*[lab={\footnotesize{S-WriteCkt}}]
               {
                 \widetilde{\rho} \vdash x \Downarrow [\overline{r_{i}}]_{n}\quad
                 \widetilde{\rho} \vdash \widetilde{e_{1}} \Downarrow n_{1}\\\\
                 n_{1} < n\quad
                 \mathsf{fresh}\:r\quad
                 \widetilde{\rho} \vdash \widetilde{e_{2}} \Downarrow \kappa\\\\
                 \widetilde{\rho'} = \widetilde{\rho}[x \mapsto [\overline{r_{i}}]_{n}[n_{1} \mapsto r]]
               }
               {
                 \widetilde{\rho} \vdash x[\widetilde{e_{1}}] := \widetilde{e_{2}} \Downarrow \widetilde{\rho'}; \kw{bind}\:\kappa\:r
               }

\end{array}
  \]
\label{fig:tgtsem}
\caption{Target semantics (selected rules)}
\end{figure}

\begin{figure}
  \small
  \fbox{$\widehat{\rho_{1}}, \widehat{\rho_{2}} \vdash \kappa \Downarrow b_{1}, b_{2}$}
  \[
  \\
  \begin{array}{c}
    %% \inferrule*[lab={\footnotesize{C-Wire}}]
    %%            {
    %%              b_{1}, b_{2} = \widehat{\rho_{1}}(r), \widehat{\rho_{2}}(r)
    %%            }
    %%            {
    %%              \widehat{\rho_{1}}, \widehat{\rho_{2}} \vdash r \Downarrow b_{1}, b_{2}
    %%            }               
    \inferrule*[lab={\footnotesize{C-In}}]
               {
                 b_{1}, b_{2} = \mathcal{E}_{m}(c)
               }
               {
                 \widehat{\rho_{1}}, \widehat{\rho_{2}} \vdash \kw{in}^{m}_{j} \Downarrow b_{1}, b_{2}
               }

               \hspace{0.3cm}
               
    \inferrule*[lab={\footnotesize{C-Coerce}}]
               {
                 \widehat{\rho_{1}}, \widehat{\rho_{2}} \vdash \kappa \Downarrow b_{1}, b_{2}\\\\
                 c = \mathcal{D}_{m_{1}}(b_{1}, b_{2})\\\\
                 b'_{1}, b'_{2} = \mathcal{E}_{m}(c)
               }
               {
                 \widehat{\rho_{1}}, \widehat{\rho_{2}} \vdash \kappa \rhd m \Downarrow b'_{1}, b'_{2}
               }
\\\\

    \inferrule*[lab={\footnotesize{C-Add}}]
               {
                 \forall i \in \{1, 2\}.\:\widehat{\rho_{1}}, \widehat{\rho_{2}} \vdash \kappa_{i} \Downarrow b_{1i}, b_{2i}\quad
                 n_{i} = \mathcal{D}_{\mathcal{A}}(b_{1i}, b_{2i})\\\\
                 b_{1}, b_{2} = \mathcal{E}_{\mathcal{A}}(n_{1} + n_{2})
               }
               {
                 \widehat{\rho_{1}}, \widehat{\rho_{2}} \vdash \kw{add}\:\kappa_{1}\:\kappa_{2} \Downarrow b_{1}, b_{2}
               }

\\\\

    \inferrule*[lab={\footnotesize{C-Gt}}]
               {
                 \forall i \in \{1, 2\}.\:\widehat{\rho_{1}}, \widehat{\rho_{2}} \vdash \kappa_{i} \Downarrow b_{1i}, b_{2i}\quad
                 n_{i} = \mathcal{D}_{\mathcal{B}}(b_{1i}, b_{2i})\\\\
                 b_{1}, b_{2} = \mathcal{E}_{\mathcal{B}}(n_{1} > n_{2})
               }
               {
                 \widehat{\rho_{1}}, \widehat{\rho_{2}} \vdash \kw{gt}\:\kappa_{1}\:\kappa_{2} \Downarrow b_{1}, b_{2}
               }

\\\\

    \inferrule*[lab={\footnotesize{C-Mux}}]
               {
                 \forall i \in \{1, 2, 3\}.\:\widehat{\rho_{1}}, \widehat{\rho_{2}} \vdash \kappa_{i} \Downarrow b_{1i}, b_{2i}\quad
                 c_{i} = \mathcal{D}_{\mathcal{B}}(b_{1i}, b_{2i})\\\\
                 c_{1} = \top \Rightarrow b_{1}, b_{2} = \mathcal{E}_{\mathcal{B}}(c_{2})\quad
                 c_{1} = \bot \Rightarrow b_{1}, b_{2} = \mathcal{E}_{\mathcal{B}}(c_{3})
               }
               {
                 \widehat{\rho_{1}}, \widehat{\rho_{2}} \vdash \kw{mux}\:\kappa_{1}\:\kappa_{2}\:\kappa_{3} \Downarrow b_{1}, b_{2}
               }

  \end{array}
  \]
  \\\\
    \fbox{$\widehat{\rho_{1}}, \widehat{\rho_{2}} \vdash \chi \Downarrow \widehat{\rho'_{1}}, \widehat{\rho'_{2}}; O$}
  \[
  \\
  \begin{array}{c}
    \inferrule*[lab={\footnotesize{C-Bind}}]
               {
                 \widehat{\rho_{1}}, \widehat{\rho_{2}} \vdash \kappa \Downarrow b_{1}, b_{2}\\\\
                 \widehat{\rho'_{1}} = \widehat{\rho_{1}}[r \mapsto b_{1}] \quad
                 \widehat{\rho'_{2}} = \widehat{\rho_{2}}[r \mapsto b_{2}]
               }
               {
                 \widehat{\rho_{1}}, \widehat{\rho_{2}} \vdash \kw{bind}\:\kappa\:r \Downarrow \widehat{\rho'_{1}}, \widehat{\rho'_{2}}; \cdot
               }
               
               \hspace{0.1cm}

    \inferrule*[lab={\footnotesize{C-Out}}]
               {
                 \widehat{\rho_{1}}, \widehat{\rho_{2}} \vdash \kappa \Downarrow b_{1}, b_{2}\\\\
                 c = \mathcal{D}_{m}(b_{1}, b_{2})
               }
               {
                 \widehat{\rho_{1}}, \widehat{\rho_{2}} \vdash \kw{out}\:\kappa \Downarrow \widehat{\rho_{1}}, \widehat{\rho_{2}}; c
               }
\end{array}
  \]
\label{fig:cktsem}
\caption{Circuit semantics (selected rules)}
\end{figure}

\begin{figure}
  \small
  \fbox{$\Gamma \vdash e : \tau \leadsto \widetilde{e}$}
  \[
  \begin{array}{c}
     \inferrule*[lab={\footnotesize{T-Cons}}]
               {
                 \tau = \mathsf{typeof}(c)^{\mathcal{P}}
               }
               {
                 \Gamma \vdash c : \tau \leadsto c 
               }

     \inferrule*[lab={\footnotesize{T-Add}}]
               {
                 \forall i \in \{1,2\}.\:\Gamma \vdash e_{i} : \kw{uint}^{\ell} \leadsto \widetilde{e_{i}}\\\\
                 \ell = \mathcal{P} \vee \ell = \mathcal{A}
               }
               {
                 \Gamma \vdash e_{1} + e_{2} : \kw{uint}^{\ell} \leadsto \widetilde{e_{1}} +_{\ell} \widetilde{e_{2}}
               }

    %% \inferrule*[lab={\footnotesize{T-Cons}}]
    %%            {
    %%              \tau = \mathsf{typeof}(c)^{\mathcal{P}}
    %%            }
    %%            {
    %%              \Gamma \vdash c : \tau \leadsto c
    %%            }

    %%  \inferrule*[lab={\footnotesize{T-Var}}]
    %%            {
    %%            }
    %%            {
    %%              \Gamma \vdash x : \Gamma(x) \leadsto x
    %%            }
\\\\
     \inferrule*[lab={\footnotesize{T-Gt}}]
               {
                 \forall i \in \{1,2\}.\:\Gamma \vdash e_{i} : \kw{uint}^{\ell} \leadsto \widetilde{e_{i}}\\\\
                 \ell = \mathcal{P} \vee \ell = \mathcal{B}
               }
               {
                 \Gamma \vdash e_{1} > e_{2} : \kw{bool}^{\ell} \leadsto \widetilde{e_{1}} >_{\ell} \widetilde{e_{2}}
               }

     \inferrule*[lab={\footnotesize{T-Read}}]
               {
                 \Gamma \vdash x : \sigma^{\ell}[n] \leadsto x\\\\
                 \Gamma \vdash e : \kw{uint}^{\mathcal{P}} \leadsto \widetilde{e}\\\\
                 \Gamma \models e < n
               }
               {
                 \Gamma \vdash x[e] : \sigma^{\ell} \leadsto x[\widetilde{e}]
               }

\\\\               

     \inferrule*[lab={\footnotesize{T-Cond}}]
               {
                 \Gamma \vdash e : \kw{bool}^{\ell} \leadsto \widetilde{e}\\\\
                 \forall i \in \{1,2\}.\:\Gamma \vdash e_{i} : \sigma^{\ell'} \leadsto \widetilde{e_{i}}\\\\
                 \ell = \mathcal{P} \vee (\ell = \mathcal{B} \wedge \ell' =\mathcal{B})
               }
               {
                 \Gamma \vdash \cond{e}{e_{1}}{e_{2}} : \sigma^{\ell'} \leadsto \lcond{\ell}{\widetilde{e}}{\widetilde{e_{1}}}{\widetilde{e_{2}}}
               }

     \inferrule*[lab={\footnotesize{T-Inp}}]
               {
               }
               {
                 \Gamma \vdash \kw{in}_{j} : \sigma^{m} \leadsto \kw{in}^{\sigma}_{j}
               }
               
\\\\               

     \inferrule*[lab={\footnotesize{T-Arr}}]
               {
                 \forall i \in \{0 \dots n - 1\}.\:\Gamma \vdash e_{i} : \sigma^{\ell} \leadsto \widetilde{e_{i}}
               }
               {
                 \Gamma \vdash [\overline{e_{i}}]_{n} : \sigma^{\ell}[n] \leadsto [\overline{\widetilde{e_{i}}}]_{n}
               }

     \inferrule*[lab={\footnotesize{T-Sub}}]
               {
                 \Gamma \vdash e : \sigma^{\ell} \leadsto \widetilde{e}
               }
               {
                 \Gamma \vdash e : \sigma^{m} \leadsto \widetilde{e} \rhd m
               }

  \end{array}
  \]
  \\\\
    \fbox{$\Gamma \vdash s \leadsto \widetilde{s} \mid \Gamma'$}
  \[
  \\
  \begin{array}{c}
     \inferrule*[lab={\footnotesize{T-Decl}}]
               {
                 \psi = \sigma \Rightarrow \tau = \sigma^{\ell}\\\\
                 \psi = \sigma[n] \Rightarrow \tau = \sigma^{\ell}[n]\\\\
                 \Gamma \vdash e : \tau \leadsto \widetilde{e}
               }
               {
                 \Gamma \vdash \psi\:x = e \leadsto \tau\:x = \widetilde{e} \mid \Gamma, x:\tau
               }

     \inferrule*[lab={\footnotesize{T-Assgn}}]
               {
                 \Gamma(x) = \sigma^{\ell}\\\\
                 \Gamma \vdash e : \sigma^{\ell} \leadsto \widetilde{e}
               }
               {
                 \Gamma \vdash x := e \leadsto x = \widetilde{e} \mid \Gamma
               }

\\\\

     \inferrule*[lab={\footnotesize{T-For}}]
               {
                 \Gamma' = \Gamma, x::\kw{uint}^{\mathcal{P}}\\\\
                 \Gamma' \vdash \loops{x}{n_{2}}{s} \leadsto \loops{x}{n_{2}}{\widetilde{s}} \mid \Gamma'
               }
               {
                 \Gamma \vdash \for{x}{n_{1}}{n_{2}}{s} \leadsto \for{x}{n_{1}}{n_{2}}{\widetilde{s}} \mid \Gamma
               }

               \\\\

     \inferrule*[lab={\footnotesize{T-Write}}]
               {
                 \Gamma \vdash x : \sigma^{\ell}[n] \leadsto x\\\\
                 \Gamma \vdash e_{1} : \kw{uint}^{\mathcal{P}} \leadsto \widetilde{e_{1}}\\\\
                 \Gamma \vdash e_{2} : \sigma^{\ell} \leadsto \widetilde{e_{2}}\\\\
                 \Gamma \models e_{1} < n
               }
               {
                 \Gamma \vdash x[e_{1}] := e_{2} \leadsto x[\widetilde{e_{1}}] := \widetilde{e_{2}} \mid \Gamma
               }

     \inferrule*[lab={\footnotesize{T-Out}}]
               {
                 \Gamma \vdash e : \sigma^{m} \leadsto \widetilde{e}
               }
               {
                 \Gamma \vdash \kw{out}\:e \leadsto \kw{out}\:\widetilde{e} \mid \Gamma
               }

\\\\

     \inferrule*[lab={\footnotesize{T-If}}]
               {
                 \Gamma \vdash e : \kw{bool}^{\mathcal{P}} \leadsto \widetilde{e}\\\\
                 \forall i \in \{1, 2\}.\:\Gamma \vdash s_{i} \leadsto \widetilde{s_{i}} \mid \_
               }
               {
                 \Gamma \vdash \ite{e}{s_{1}}{s_{2}} \leadsto \ite{\widetilde{e}}{\widetilde{s_{1}}}{\widetilde{s_{2}}}  \mid \Gamma
               }

     \inferrule*[lab={\footnotesize{T-Seq}}]
               {
                 \Gamma \vdash s_{1} \leadsto \widetilde{s_{1}} \mid \Gamma_{1}\\\\
                 \Gamma_{1} \vdash s_{2} \leadsto \widetilde{s_{2}} \mid \Gamma_{2}
               }
               {
                 \Gamma \vdash s_{1}; s_{2} \leadsto \widetilde{s_{1}}; \widetilde{s_{2}} \mid \Gamma_{2}
               }

\\\\

     \inferrule*[lab={\footnotesize{T-While}}]
               {                 
                 \Gamma(x) = \kw{uint}^{\mathcal{P}}\quad
                 \Gamma \vdash s \leadsto \widetilde{s} \mid \_\quad
                 x \notin \mathsf{modifies}(s)
               }
               {
                 \Gamma \vdash \loops{x}{n_{2}}{s} \leadsto \loops{x}{n_{2}}{\widetilde{s}} \mid \Gamma
               }

  \end{array}
  \]
\label{fig:compile}
\caption{Compilation judgments (selected rules)}
\end{figure}

  %% \\
  %% \[
  %% \begin{array}{rrcl}
  %%   \ftext{Value} & v &::=& c \mid [\overline{c_{i}}]_{n}\\
  %%   \ftext{Observation} & O &::=& \cdot \mid c; O\\
  %% \end{array}
  %% \]

\divya{Start here......}
\subsubsection*{Security theorem} The protocols we generate provide
simulation-based security against a semi-honest adversary, in the
framework of ~\cite{gmw,can00,can01}. At a very high level, in this
framework, parties are modeled as non-uniform interactive turing
machines (ITMs), with inputs provided by an environment $\env$. An
adversary $\adv$, selects and ``corrupts'' one of the parties -
however, $\adv$ still follows the protocol specification. $\adv$
interacts with the $\env$, which observes the view of the corrupted
party. At the end of the interaction, $\env$ outputs a single bit. Two
different interactions are defined: the {\em real world} and an {\em
  ideal world}. In the real interaction, the parties run the protocol
$\prot$ in the presence of $\adv$ and $\env$. Let
$\real_{\prot,\adv,\env}$ denote the binary distribution ensemble
describing $\env$'s output in this interaction. In the ideal
interaction, parties send their inputs to an additional entity, a
trusted functionality machine $\F$ that carries the desired
computation truthfully. Let $\simu$ (the simulator) denote the
adversary in this idealized execution, and $\ideal_{\F,\simu,\env}$
the binary distribution ensemble describing $\env$'s output after
interacting with adversary $\simu$ and ideal functionality $\F$. A
protocol $\prot$ is said to {\em securely realize} a functionality
$\F$ if for every adversary $\adv$ in the real interaction, there is
an adversary $\simu$ in the ideal interaction, such that no
environment $\env$, on any input, can tell the real interaction apart
from the ideal interaction, except with negligible probability (in the
security parameter $\secparam$). More precisely, if the two binary
distribution ensembles above are computationally indistinguishable. 

We shall assume a cryptographic MPC backend that securely implements any circuit $\chi$ that is output by our compiler. In more detail, this means that for every source program $s$, let $\chi$ be the circuit output by our compiler (as in Theorem \ref{theorem:correctness}). We assume that there exists a two-party secure computation protocol $\prot$ that securely realizes the functionality $\chi$ and we will call the corresponding simulator for the protocol as $\simu_{\tiny{2pc}}$ (that runs on input $\chi$ and the output produced by $\chi$ on the parties inputs). We note that the work of \cite{aby} provides such a protocol $\prot$ and simulator $\simu_{\tiny{2pc}}$ for all circuits $\chi$ output by our compiler. We are now ready to state and prove our security theorem.
 
\begin{theorem}[Security]\label{theorem:security}
Let $s$ be any functionality or program in our source language with outputs (or observations) $O_1$, that is compiled into a circuit $\chi$ (as defined in Theorem \ref{theorem:correctness}). Let protocol $\prot$ be the two-party secure computation protocol that securely realizes $\chi$ (as defined above). Then, $\prot$ securely realizes $s$.  
\end{theorem}

\noindent {\em Proof.} Our simulator $\simu$ simply runs our compiler on program $s$ to obtain $\chi$ and upon receiving output $O_1$, executes $\simu_{\tiny{2pc}}$ on $\chi$ and $O_1$. Let $O_2$ be the outputs (or observations) of $\chi$. First, from Theorem \ref{theorem:correctness}, we have that the observations in $s$ and $\chi$ are the same (i.e. $O_2 = O_1$). Now, from the security of $\prot$, we have that the simulated view output by $\simu_{\tiny{2pc}}$ (with circuit $\chi$ and output $O_2$) is indistinguishable from the real view of the $\prot$ (when executed on circuit $\chi$ and when the output is $O_2$). Combining these two statements, the proof of the theorem follows. 
\aseem{Can we give a construction of the simulator S in the ideal
  semantics? The simulator S whenever it receives an observable event,
  outputs it to Z, and whenever A receives a string in the real
  semantics, S outputs a random string to Z. Now, with theorem 1,
  observable events are same in the ideal and real semantics, and with
  the security of the crypto backend, the random strings and the
  strings in the protocol are indistinguishable. Is it
  incorrect/naive?

  Also, we should emaphasize here that the ideal semantics is the
  source semantics from Figure x and the real semantics is the circuit
  semantics from Figure y. For the target language semantics, we
  should emphasize that it does not have access to the secrets, so
  there are no observations there.
}

\nc{OK now?}
  %% \\
  %% \[
  %% \begin{array}{rrcl}
  %%   \ftext{Value} & v &::=& c \mid [\overline{c_{i}}]_{n}\\
  %%   \ftext{Observation} & O &::=& \cdot \mid c; O\\
  %% \end{array}
  %% \]

\section{Secure code partitioning}
\label{sec:pipe}

We  describe ``secure code partitioning'' that allows us to execute {\it large} programs, i.e., programs with a large number of operations.
In more detail, let $P$ be a program in our source language that generates a circuit $\crct$. For a large enough $P$, the circuit $\crct$  can be larger than the memory size\footnote{In fact, there is an upper limit of $2^{32}-1$ gates for the circuit size in ABY but for most machines the memory limit is hit first.} and fail to execute. Partitioning enables us to
execute such programs via a source to source transformation that is oblivious to the underlying \mpc backend. Partitioning decomposes the program $P$ into a sequence of smaller \tool programs $Q_1||Q_2||\cdots||Q_k$ (defined below). We compile and execute each $Q_i$ sequentially, feeding the outputs of $Q_i$ as state information to $Q_{i+1}$. We prove that partioning is correct ($P$ and $Q_1||Q_2||\cdots||Q_k$ compute the same functionality) and secure (execution of $Q_1||Q_2||\cdots||Q_k$ does not reveal any more information than $P$). More details follow.


Let $P$ be a program that takes (secret) inputs $x$ from Alice and $y$ from Bob and produces an output $z$ to both parties. Let $P_1||P_2||\cdots||P_k$ be a sequence of programs such that the following holds: Define $s_0 = \bot$ (the public empty state). For all $1\leq i\leq k-1$, $P_i$ takes inputs $x, y$ and $s_{i-1}$ and outputs  state information $s_i$. Finally, $P_k$ takes inputs $x,y$ and $s_{k-1}$ to  output $z$. Each program $P_i$ ($1\leq i\leq k$) is required to be ``sufficiently'' small (as described later). It is possible to decompose any program $P$  into such $P_1||P_2||\cdots||P_k$. If \tool generates circuit $\crct_i$ from $P_i$, the parties can execute $\crct_1||\crct_2||\cdots||\crct_k$ sequentially (in a distributed setting)  to obtain $s_1,\cdots,s_{k-1},$ and finally output $z$. At the $i^{\textrm{\tiny{th}}}$ step, the parties only need to store information proportional to $x,y,s_{i-1}$ and $\crct_i$ (which is much smaller than $\crct$). However, this execution enables the parties to learn $s_i$ (for all $1\leq i\leq k-1$), which completely breaks the security.

To overcome this problem, we define a sequence of new programs $Q_i$ ($1\leq i\leq k$) as follows. Once again, define $s_0 = \bot$. Without loss of generality, let all $s_i$ be values in some additive ring $(\mathbb{Z},+)$ (e.g., the additive ring $(\mathbb{Z}_{2^{64}},+)$, i.e., the additive ring of integers modulo $2^{64}$). 
Let $r_1,\cdots,r_{k-1}$ be a sequence of random values sampled from the same ring $(\mathbb{Z},+)$ by Alice (in our implementation, all $r_i$ values are generated by a pseudorandom function). Let $Q_1$ be the program that takes as input $x,r_1$ from Alice and $y$ from Bob (and empty state $s_0$), and runs $P_1$ (as defined above) to compute $s_1$ and then outputs $o_1 = s_1 + r_1$ {\em only to} Bob\footnote{While the description of our protocol here assumes that the underlying backend supports only one party receiving output, this is only a simplifying assumption, and we can easily modify our protocol in the case where both parties must receive the same output. To do so, we modify $Q_1$ to output $o_1 = s_1+r_1+t_1$ to both parties (where $t_1$ is random and chosen by Bob). We can then appropriately modify the remaining steps as well.}. Alice's output from $Q_1$ is $r_1$. Next, every $Q_i$ ($2\leq i\leq k-1$) takes as inputs $x,r_{i-1},r_i$ from Alice and $y,o_{i-1}$ from Bob, runs $P_i$ on inputs $x,y$ and state $o_{i-1}-r_{i-1}$ (where $-$ denotes subtraction in the ring $(\mathbb{Z},+)$) and then outputs $s_i+r_i$ only to Bob. Again, Alice's output from $P_i$ is $r_i$. The last program $Q_k$ takes inputs $x,y,r_{k-1},o_{k-1}$, runs $P_k$ on inputs $x,y$ and state $o_{k-1}-r_{k-1}$ and outputs $z$ to both parties.

Given $P$, partitioning first decomposes $P$ into $P_1||P_2||\cdots||P_k$ with the guarantee that the circuits generated for every $Q_i$ defined above are small enough to fit in the machine memory. We then compile and execute the $k$ programs $Q_1||Q_2||\cdots||Q_k$ sequentially (freeing up memory usage after  execution of each $Q_i$) to finally output $z$ to both parties.

\begin{theorem}[Correctness and security of partitioning]
Let $P$ be a program in our source language with output $z$, that is decomposed into programs $Q_1||Q_2||\cdots ||Q_k$ (as defined in the partioning procedure above) and compiled into the protocol $\prot = \prot_1||\prot_2\cdots ||\prot_k$ that outputs $r_1||r_2||\cdots||z$ (to Alice) and $o_1||o_2||\cdots||z$ (to Bob), then $\prot$ securely realizes $P$.  
\end{theorem}

\noindent {\em Proof.} From \theoremref{security}, it follows that $\prot_i$ securely realizes $Q_i$ for every $1\leq i\leq k$. Hence, we know that the only information learnt by Alice is $r_1||r_2||\cdots||r_{k-1}||z$ and by Bob is $o_1||o_2||\cdots||o_{k-1}||z$. Since $r_i$ and $o_i$ ($1\leq i\leq k-1$) are individually uniformly random (in $(\mathbb{Z},+)$, outputs received by the adversary can be easily simulated given the final output $z$. 

Our current implementation of secure code partitioning requires a manual step.
In particular, the developer needs to decompose the large program $P$ into 
the sequence of small programs $P_1||\ldots||P_k$ manually.
Then \tool generates $Q_1||\ldots||Q_k$ automatically.
Automating this decomposition step requires an analysis that can 
statically estimate the resource usage of a \tool program and we leave the construction of such
an analysis as future work.
\vspace{-0.1in}
\section{Implementation}
\label{sec:impl}
We discuss some  implementation details of \tool.
The  \tool compiler is written in Python and
compiles each of our benchmarks in under a second to C++ code that
makes calls to the ABY
library~\cite{aby}. 
ABY provides support for Arithmetic computation based on \cite{autoS}, and boolean computations based on GMW~\cite{gmw} as well as Yao's garbled circuits~\cite{yao}. Although \tool can generate code for both kinds of boolean computations, we have observed better performance when using the garbled circuits
and use it in our evaluation. Hence, \tool generated code uses Arithmetic computation and garbled circuits based boolean computations.
We use $128$ bits of security and OT
extension-based arithmetic multiplication triplets generation. ABY
provides multi-threading support (for the offline phase of the \mpc
protocol); we leverage the support and use at most four threads in our
evaluation. 

\tool programs can have the following operators:
addition, subtraction, multiplication, division by powers of two, left
shift,
logical and arithmetic right shifts, bitwise-(and, or, xor), unary
negation, 
bitwise-negation, logical-(not, and, or, xor), and comparisons (less
than, greater than, equality).  
Because of their high cost, integral division and floating-point
operators are not supported natively by \tool. 
However, we have implemented integral division in 30 lines of \tool,
while the floating-point support in ABY is under active
development~\cite{ddkssz15}.

Some of our benchmarks require accessing arrays at secret
indices. While \tool enforces the array indices to be
public, secret indices can be encoded in \tool using multiplexers.
For example, consider the expression $A[x]$ where $A$ is an array of
size 2 and $x$ is secret-shared. The developer can express
this functionality in \tool as $\cond{x > 0}{A[1]}{A[0]}$. In general,
a secret access to an array of
size $n$ requires ${n}-1$ multiplexers in \tool.

%Moved this to Section 4 since P_i and Q_i are ill defined here
%Secure code partitioning currently requires a manual step.
%In particular, the developer needs to decompose a program $P$ into
%$P_1||\ldots||P_k$ manually (Section~\ref{sec:pipe}). Then \tool
%generates $Q_1||\ldots||Q_k$ (the sub-programs with appropriate
%wrappers) automatically.
%Automating the decomposition step requires an analysis that can 
%statically estimate the resource usage of an \tool program. We leave
%the construction of such an analysis as future work.

We use an off-the-shelf solver
(SeaHorn~\cite{seahorn}) to bound check the array indices
($\models e < n$
in ({\sc {T-Read}}) and ({\sc{T-Write}}),
\figureref{compile}). We take the \tool source program and
translate it as an input C program to the solver. The solver takes less
than a minute on our largest benchmark to verify that all the array
accesses are in-bounds. This C program can also enable
validation of \tool generated protocols via differential testing~\cite{mckeeman,frigate}.

Our implementation assigns the type labels (rule~{\sc{T-Decl}})
conservatively. Only the variables that govern the control flow, i.e.,
variables in \kw{if}-conditions and \kw{for}-loop counters are
assigned public labels.
All other variables are assigned arithmetic labels (that can later be
coerced to boolean).
We leave a more sophisticated type inference procedure for future work.

The compilation rules of \figureref{compile} can introduce
repeated coercions from arithmetic to
boolean and vice versa.
Since \tool is aware of the cryptographic costs associated with these coercions,
it tries to minimize them using several optimizations.
For instance, if boolean shares of a variable are available in scope and the variable is involved in
an addition then \tool performs the addition in boolean rather than first coercing boolean to arithmetic
and then performing an arithmetic addition. 
%However, multiplication is always performed in arithmetic as the cost of a boolean multiplication is much higher than the cost of coercing from boolean to arithmetic, performing an arithmetic multiplication, and then coercing from arithmetic to boolean~\cite{aby}.
Other optimizations include the standard ``common subexpression elimination"
optimization~\cite{dragonbook}.
On each coercion, \tool memorizes the pair of arithmetic
share and boolean share involved in the coercion. 
\tool invalidates such pairs when the variables corresponding to the
shares are overwritten by assignments. 
In subsequent coercions, \tool  reuses valid pairs (if available)
instead of inserting code to recompute them afresh.

All these optimizations are standard compiler
optimizations~\cite{dragonbook}, and we rely on their correctness to
maintain the correctness and security for the optimized programs.



We first implement benchmarks of ~\cite{shafindss} and~\cite{minionn} in \tool.
Next, we show the generality of \tool by implementing state-of-the-art machine learning models
. 
lin
nb
tree\\
minion-cnn
secureml
cryptonet\\
logistic
cnn
bonsai\\
cifar
\subsection{Matrix factorization}
\tool, as a language, is not tied to secure prediction and can express more general computations.
To demonstrate the expressiveness of \tool, we use it to implement secure matrix factorization~\cite{valeriaMatrix}. Abstractly, given a sparse matrix $\mathcal{M}$ of dimensions
$n\times m$ and $M$ non-zero entries, the goal is to generate a matrix $U$ of dimension $n\times d$ and a matrix
$V$ of dimension $d\times m$ such that $\mathcal{M}\approx UV$. This operator is useful in recommender systems.
In particular, Nikolaenko et al.~\cite{valeriaMatrix} shows how to implement a movie recommender system which does not require users to reveal their data in the clear, i.e., the ratings the users have assigned to movies. The implementation is a two party computation of an iterative algorithm for matrix factorization (Algorithm 1 in~\cite{valeriaMatrix}).
This algorithm is based on gradient descent and iteratively converges to a local minima.
We implement this algorithm in \tool.
  
To ensure that the algorithm converges to the right local minima, Nikolaenko et al. require
36 bits of precision. Since ABY supports either 32-bit or 64-bit integers, our \tool implementation
manipulates 64-bit variables. For $\mathcal{M}$ Nikolaenko et al. consider $n=940$ users, $m=40$ most popular movies, and $M=14683$ ratings from the MovieLens dataset. The time reported in~\cite{valeriaMatrix}
for one iteration is 2.9 hours\footnote{~\cite{valeriaMatrix} does not report the network round-trip time.}. This compuation is large enough that we partition each iteration
into three stages. The first stage involves a Batcher~\cite{Batcher} sorting network followed by a linear pass.
The second stage involves sorting and gradient computations and is the heaviest stage.
The third  stage is similar to the first stage. The results are presented in Table~\ref{tab:factor}.


\begin{table}
\begin{tabular}{c|c|c|c |c|c| c}
  Stage         &  LAN (s) & WAN (s) & Comm. (Mb)  & depth & \#Gates & LOC\\
\hline
1    &  175       & 662        & 29816       & 16370    & 33m    & 500  \\
\hline
2    &  193        & 1095        & 31945        & 30916    & 37m & 516 \\
\hline
3    &  178        & 627        & 29810        & 16369    & 32m  & 478  \\
\hline
Total    &  546      & 2384        & 91571        & --    & 102m & 1494 \\
\end{tabular}

 \caption{Partitioning results for matrix factorization. The time reported by~\cite{valeriaMatrix} for this computation is about 10440 seconds.}
 \label{tab:factor} 
\end{table}

We observe that in the LAN setting, we are about 19 times faster than~\cite{valeriaMatrix} and in the WAN
setting we are about 4 times faster. The main source of these significant speedups is that, unlike~\cite{valeriaMatrix}, \tool does not need to convert the functionality into boolean circuits. 
However, this benchmark requires more lines of code than the previous benchmarks.
It is inconvinient to write Batcher's sort in an iterative language (about 450 lines of \tool).
The size of a recursive implementation would be much slower.
In the future, we would like to add support for functions and recursion.
However, the current programmer effort seems miniscule compared to the mammoth implementation effort
put in by Nikolaenko et al. (Section 5 of ~\cite{valeriaMatrix}) to scale a boolean circuit based
crypto-backend to this benchmark.

\section{Related work}
\label{sec:related}

%\divya{Related Work: writing some citations here to be used appropriately.}
%\begin{enumerate}
%\item Secure Computation of MIPS Machine Code https://eprint.iacr.org/2015/547
%\item TASTY https://eprint.iacr.org/2010/365.pdf. Does not have language; user has to know what parts in HE and what in Yao and writes something similar to ABY
%\item ABY cites many papers on first page of intro for using a mix of HE and boolean MPC; but they didnt give much gain because HE is expensive
%\nc{I will look into this and write.}
%\item verilog, hardware tool chain paper http://encrypto.de/papers/DDKSSZ15.pdf
%\item automatic choosing between HE and Yao https://eprint.iacr.org/2014/200.pdf 
%I think this one still does not have a language; I don't know how to specify function
 
\tool falls into the category of frameworks that compile high level languages to \mpc protocols. We discuss other such frameworks next.
%Existing \mpc tools work with either a high or a low level desciption of the functionalities to be implemented securely.
 %\end{enumerate}
%\noindent\textbf{Generating \mpc protocols.}
%There has been vast literature on the generation of \mpc protocols. These works all provide tools at various levels of abstraction that help a programmer generate cryptographic protocols for various functionalities. These levels can be classified as follows: 
%Examples of tools in the former category include the following: 
Fairplay's Secure Function Definition Language (SFDL)~\cite{fairplay,fairplaymp} and CBMC-GC~\cite{cbmcgc} compile C or Pascal like programs into boolean circuits that are then evaluated using garbled circuits~\cite{yao}.
ObliVM~\cite{oblivm} protects access patterns using an oblivious RAM~\cite{oram1,oram2} and also uses garbled circuits for compute.
In Secure Multiparty Computation Language (SMCL)~\cite{smcl},  Java
like programs are compiled into arithmetic circuits that are then
evaluated using the VIFF
framework~\cite{viff}. Wysteria~\cite{wysteria} enables programmers to
write $n$-party mixed-mode programs that combine local, per-party
computations with secure computations. It compiles secure
computations to
boolean circuits and uses a GMW-based backend~\cite{choi,gmw}.
 Mitchell et al.~\cite{lambdaps} allow the user to select between
 Shamir's secret sharing~\cite{sss} and fully homomorphic
 encryption~\cite{gentry}.  Unlike \tool, all these tools use either
 an arithmetic backend or a boolean backend but not a combination of
 both.

Next, we discuss tools that expose libraries which  developers can use to describe \mpc protocols. 
% The exact cryptographic protocols that are used by the library  need not be known to the programmer. However, in order 
To generate efficient protocols for a functionality, the programmer must break the functionality into components and call the appropriate library functions. For example,  ABY \cite{aby} falls in this category. The TASTY tool~\cite{tasty} allows mixing homomorphic encryption based arithmetic computations 
and garbled circuits based boolean computations and the interconversions between the two are inserted by the programmer explicitly.
% While, the programmer need not know how ABY implements boolean or arithmetic computation, he or she must know the type of compute (boolean or arithmetic) that is more efficient for specific sub-computations and also explicitly write calls to convert shares from one scheme to another (boolean secret sharing to arithmetic secret sharing and vice-versa). 
The work of Kerschbaum et al.~\cite{autoS} provides a scheme to automatically assign homomorphic encryption or garbled circuits to each operator in a computation that is expressed as a sequence of dyadic operations. They conjecture that the problem is NP hard and gave a linear programming based solution and a quadratic time greedy heuristic. 
These techniques are not directly applicable to \tool programs because of $\mathtt{for}$-loops and $\mathtt{if}$-conditions.
However, we are exploring if these ideas can be extended to yield better type inference.
%Their performance is much poorer than ABY due to the high 
%costs associated with homomorphic encryption~\cite{aby}.
%conversion costs between homomorphic encryption and boolean shares~\cite{aby}.  %On the other hand \tool uses a simpler constant time heuristic to assign arithmetic secret sharing or garbled circuits to sub-computations in a richer language with control flow statements such as loops and branches.
%While the work of Kerschbaum {\em et al.}~\cite{kos14} provides a heuristic to automatically select the protocol for various types of sub-computations, one would still need to manually write the program using the share conversion tools of ABY. Similarly, the TASTY tool~\cite{tasty} provides a way to mix both boolean and arithmetic circuits (using homomorphic encryption for the arithmetic compute). 
Other examples include the VIFF framework~\cite{viff} for arithmetic computations and Sharemind~\cite{sharemind} (secure 3-party boolean computation). 
%\\\\
%\noindent{\bf Protocol Implementation:} Tools at this level of abstraction require the programmer to have a deep understanding of how cryptographic secure computation protocols work and the programmer must provide calls to low-level cryptographic building blocks such as oblivious transfer. For example while the programmer need not know the specific oblivious transfer protocol implemented in the backend, he or she still needs to know how to build a secure computation protocol from such building blocks. This abstraction requires the highest knowledge of cryptography from the programmer. Examples of such tools include L1~\cite{lone} and Qilin~\cite{qilin}.

\mpc backends  have made  tremendous progress in the last decade.
For example, the circuits can be optimized for depth~\cite{ddkssz15,cbmcgcdepth}, large garbled circuits can be pipelined~\cite{yao-pipe,oblivm}, online complexity can be reduced at the cost of offline complexity~\cite{groce}, encrypted values output from a garbled circuit can be reused~\cite{ReuseOrLose} and oblivious RAM~\cite{oram1,oram2} can be used to hide access patterns of MIPS code~\cite{mips}. 
Incorporating these backends would only improve the performance and scalability of \tool implementations. 

%\\\\
%\noindent\textbf{Custom \mpc protocols.} 
Many works have designed specialized protocols for various \mpc tasks. 
%The reason for doing this was to allow for a mix between arithmetic and boolean computations as a secure computation protocol for a pure boolean circuit would have been very inefficient. 
This requires deep knowledge of cryptography
to ensure security.
 Examples include ~\cite{barni,blanton,brickell,franz,huang,valeriaMatrix, valeriaRidge,shafindss,wu,secureml,minionn}.

%Examples include classification of medical ElectroCardioGram (ECG)~\cite{barni}, iris and fingerprint recognition~\cite{blanton}, remote diagnostics~\cite{brickell}, sequence analysis~\cite{franz}, biometric identification~\cite{huang}, matrix factorization and ridge-regression~\cite{valeriaMatrix, valeriaRidge}, machine-learning classification~\cite{shafindss}, decision trees/forests~\cite{wu}, neural network training~\cite{secureml} and prediction~\cite{minionn}.

% As seen from our experiments, the protocols generated by \tool give us better or comparable execution times as many of the above works with minimal programmer overhead. %\nc{What do we want to say here exactly?}

\section{Conclusion}
\label{sec:conclude}
We presented \tool, the first
cryptographic-cost aware framework that generates efficient and
scalable \mpc protocols from high-level programs.
The generated protocols comprise combinations
of arithmetic and boolean circuits and have performance comparable to, or better than the custom, specialized
protocols from previous works.
The \mpc backed we use provides security against semi-honest or passive adversaries.
%our compiler is oblivious to the ``type of adversary'' for the underlying \mpc backend.
Given a \mpc backend secure against malicious or active adversaries, Theorem \ref{theorem:security} 
can be extended to obtain a framework with malicious security. Furthermore,  
partitioning can be modified to provide malicious security by {\it signing} the shares of the intermediate states. However, we are not aware of a maliciously secure \mpc implementation for combinations of
arithmetic and boolean circuits.
\\
\\
\\
\\
\\
\\
% Several areas of improvement remain: a) While the backend that we use~\cite{aby} only provides security against a semi-honest adversary, our compiler is oblivious to the ``type of adversary'' for the underlying backend cryptographic protocol. In particular, if our cryptographic backend were maliciously secure, then Theorem \ref{theorem:security} would provide us a framework with malicious security. Furthermore, modifying our code partitioning to provide malicious security is fairly easy (one would instead provide {\em signed} shares of intermediate states in the decomposed programs to obtain malicious security). b) Our compiler converts programs into circuits before executing a \mpc and is efficient for the applications we present. However, one could consider executing programs that exploit secret-dependent memory accesses (such as binary search) to obtain better performance (as in \cite{oblivm}). It would be interesting to see if our techniques can be combined with that of \cite{oblivm}. \nc{Too much? Should we simply say only about the malicious security and stop?}


%\newpage

{\footnotesize \bibliographystyle{acm}
\bibliography{bib}}


%\bibliographystyle{abbrv}
%\bibliography{bib}
%\balance

\appendix

%\newpage

\setlength\intextsep{5pt}
\setlength{\textfloatsep}{5pt}
\setlength{\abovecaptionskip}{5pt}
\setlength{\belowcaptionskip}{5pt}

\section{Example of secure code partitioning}\label{app:codepartitioning}
We now illustrate code partitioning through an example. Consider the functionality in Figure \ref{fig:largecode}. This is a functionality that takes as input two vectors $w$ and $v$ from Alice and two vectors $x$ and $y$ from Bob. It computes two inner products $w^Tx$ and $v^Ty$, compares the first value with the second and returns a boolean value (which is $1$ if $w^Tx>v^Ty$ and $0$ otherwise) to Bob. Now, if we wish to partition this functionality using secure code partitioning, one possible split is as follows into the following three programs\footnote{All arithmetic is over an appropriate ring in the following discussion.}. Partition $1$ (Figure \ref{fig:codepartitioning1}) computes $w^Tx$ and ``secret shares'' the output of this computation between Alice and Bob (Alice's share is $r_1$, a random value, and Bob's share is $o_1 = w^Tx+r_1$). Next, partition $2$ (Figure \ref{fig:codepartitioning2}) computes $v^Ty$ and once again provides Alice with $r_2$ and Bob with $o_2 = v^Ty+r_2$. Finally, partition $3$ (Figure \ref{fig:codepartitioning3}) compares $o_1-r_1$ with $o_2-r_2$ and provides the output to Bob. It is easy to see that the size of the programs $1, 2$ and $3$ (and their corresponding circuits output by the \tool compiler) are smaller than the program in Figure \ref{fig:largecode} and its corresponding circuit, and in particular, smaller than the state that must be maintained between the programs.

\begin{figure}
\begin{lstlisting}[language=C,mathescape=true]
$\mathtt{
uint\;w[30] = input1();\;\;uint\;v[30] = input1();
}$
$\mathtt{
uint\;x[30] = input2();\;\;uint\; y[30] = input2();
}$
$\mathtt{
uint\;acc1 = 0;\;\;uint\;acc2 =0;
}$
$\mathtt{
for\;i\;in\;[0:30]\;
}$
$\mathtt{\{acc1\;=\;acc1\;+\;(w[i]\;\times\;x[i]);\;
}$
$\mathtt{\;\;acc2\;=\;acc2\;+\;(v[i]\;\times\;y[i]);\}
}$
$\mathtt{
output2((acc1 > acc2)\;?\;1\;:\;0)\;\textbf{{\color{blue}//only to party 2}}
}$
\end{lstlisting}
\caption{\tool code for $w^Tx > v^Ty$}
\label{fig:largecode}
\end{figure}

\begin{figure}
\begin{lstlisting}[language=C,mathescape=true]
$\mathtt{
uint\;w[30] = input1();\;\;uint\;r1 = input1();
}$
$\mathtt{
uint\;x[30] = input2();
}$
$\mathtt{
uint\;acc1 = 0;
}$
$\mathtt{
for\;i\;in\;[0:30]\;\{acc1\;=\;acc1\;+\;(w[i]\;\times\;x[i]);\}
}$
$\mathtt{
uint\;o1 = acc1+r1;
}$
$\mathtt{
output2(o1)\;\textbf{{\color{blue}//acc1 is ``secret shared''}}
}$
\end{lstlisting}
\caption{Partition 1: Code for $o_1 = w^Tx+r_1$}
\label{fig:codepartitioning1}
\end{figure}

\begin{figure}
\begin{lstlisting}[language=C,mathescape=true]
$\mathtt{
uint\;v[30] = input1();\;\;uint\;r2 = input1();
}$
$\mathtt{
uint\;y[30] = input2();
}$
$\mathtt{
uint\;acc2 =0;
}$
$\mathtt{
for\;i\;in\;[0:30]\;\{acc2\;=\;acc2\;+\;(v[i]\;\times\;y[i]);\;\}
}$
$\mathtt{
uint\;o2 =acc2+r2;
}$
$\mathtt{
output2(o2)\;\textbf{{\color{blue}//acc2 is ``secret shared''}}
}$
\end{lstlisting}
\caption{Partition 2: Code for $o_2 = v^Ty+r_2$}
\label{fig:codepartitioning2}
\end{figure}

\begin{figure}
\begin{lstlisting}[language=C,mathescape=true]
$\mathtt{
uint\;r1 = input1();\;\;uint\;r2 = input1();
}$
$\mathtt{
uint\;o1 = input2();\;\;uint\; o2 = input2();
}$
$\mathtt{
uint\;acc3 = o1-r1;\;\;uint\;acc4 =o2-r2;
}$
$\mathtt{
output2((acc3 > acc4)\;?\;1\;:\;0)\;\textbf{{\color{blue}//only to party 2}}
}$
\end{lstlisting}
\caption{Partition 3: Code for $(o_1-r_1)>(o_2-r_2)$}
\label{fig:codepartitioning3}
\end{figure}


\newpage

\section{Description of benchmarks}\label{app:benchmarks}
We use $[N]$ to denote $\{0,1,\dotsc, N-1\}$. Further, given a vector
$x\in\R^d$, we say $\mathsf{argmax}\ x = i$ if $x_i =
\mathsf{max}\ \{x_0,\ldots,x_{d-1}\}$. Finally, if $A$ is
a matrix (resp. vector) then we write $f(A)$ for the matrix (resp. vector)
obtained by applying the scalar function $f$ to each entry of $A$
pointwise.

We focus on the machine
learning models for {\it classification}. A classifier $C$ uses a
trained model to {\it predict} a
label $\ell$ for an input data point $x$. For example, given a
data point which is a tuple of humidity and temperature 
a classifier can predict a label ``will rain'' or ``will not rain''.
The {\it model size} of a classifier is the number of parameters in the model.
For example, the model size of the classifier in~\figureref{ex-sml} is $|w|+1=31$.
The {\it
  accuracy} of a classifier refers to the fraction of data points that
the classifier labels correctly from a given set of test data
points.

\paragraph{Standard classifiers.}
A {\em binary linear classifier} is one of the simplest classifiers. Here,
the input is a data point $x\in\R^d$,
and the model is a vector $w\in\R^d$.
 The possible labels are
$\ell\in\{\mathit{true},\mathit{false}\}$ and the classifier is
$C_w\equiv w^Tx>0$.
%, where $w^T$ denotes the transpose
%of matrix $w$, $\matmul$ denotes matrix multiplication (with a vector in this case) and $a>b$ returns %$\mathit{true}$ if $a>b$ and $\mathit{false}$ otherwise. 
%The {\it model size} of this classifier, i.e., the number of parameters in the model is $|w|$, i.e., $d$.
%This classifier requires $d$ multiplications, $d-1$
%additions, and a single comparison.
%
A more interesting classifier is {\em Na\"{i}ve Bayes}~\cite{shafindss} that predicts labels
from the set $[n]$.
Here, the input data point is a {\it feature}
vector $x=(x_0,x_2,\ldots,x_{d-1})^T$ where each $x_j\in [F]$.
%The model has two matrices: a vector $P$ of length $n$ s.t. for each
%$i
%\in [n]$, $P(i)$ is the log-probability that the output label is $i$.
%$P(i) \equiv \log p(\ell=i)$, where $p(\ell=i)$ is the likelihood
%that the output label is $i$ for $i\in[n]$.
%The other matrix $T$ has size $d\times F\times n$ and for each $j\in
%[d], k\in [F], i\in [n]$, the entry $T(j)(k)(i)$ is 
%\equiv\log p(x_j|\ell=i)$, i.e.,
%the log-probability that  the $j^{th}$ input feature is $k$  conditioned on  the output label being $i$.
%The classifier $C_{P,T}(x)$ outputs
%\[
%{\sf argmax}\ P+\sum_{j=0}^{d-1} T(j)(x_j)
%\]
%The additions in the expression above sum $n$-dimensional vectors.
The model size of this classifier is $\Theta(ndF)$.
% comparisons and additions. %Note here  that computing $T(j)(x_j)(i)$ requires a secret look-up in matrix  $T$  based on secret input $x_j$ resulting in $F-1$ comparisons using multiplexers (see
  %Section~\ref{sec:impl}). %\nc{Made DG's comment as part of regular text - worth point out}. \cmmt{Is there a +1 etc missing or is this exact?} \nc{By our calculation in Sec 5, it seems that this should be $F/2$ multiplexers}
%
A {\em decision tree} of size $N$ and depth $d$ takes as input an $x\in\R^d$
%,
 and 
%the
%model consists of a binary tree of depth $d$ with a boolean predicate
%assigned to each internal node.
% The root note is the $0^{th}$ node and
%for an internal node $j \in [N]$, the children nodes are $2j+1$ and
%$2j+2$. Each internal node $j\in[N]$ at depth $i$ has a predicate
%$b_{j}^i\equiv x_i\leq w_{j}$. We start evaluating the tree from the
%root and if the predicate at the current node $b_j^i$ is false (resp.,
%true) then $x$ is passed to the left (resp., right) child with
%predicate $b_{2j+1}^{i+1}$ (resp., $b_{2j+2}^{i+1}$). This process is
%repeated till we reach a leaf. The leaves of the tree are labels and
%the output label is the leaf visited by this traversal.
%Such a (binary) decision tree 
%and can be encoded as a polynomial (of
%degree linear in depth of the tree) and then
 the prediction task
reduces to evaluation of a $d$-degree polynomial~\cite{shafindss}.
% E.g., a binary tree with depth one
%and size $N=3$ can be encoded as the polynomial
%$(1-b(x))\ell_0 + b(x)\ell_1$ where $b(x)$ is the predicate at the
%root, $\ell_0$ is the label of the left leaf and $\ell_1$ is the label
%of the right leaf.

\paragraph{Deep neural nets.}
The next class of classifiers that we benchmark are deep neural nets
or DNNs. A DNN has multiple layers such that each layer computes a
matrix
multiplication followed by an {\it activation} function $f$. The most
common activation functions are square $f(x)=x^2$ and rectifier linear
unit (ReLU) $f(x)=\mathsf{max}(x,0)$.
Given an input vector $x$, the predicted label of a DNN is
\[
 \mathsf{argmax}\ W_N\matmul f_{N-1}(\ldots f_1(W_1\matmul x)\ldots)
\]
Here, $f_i$'s are the (public) activation functions, the model
consists of matrices $W_i$,  $x \in \R^d$ is the input vector, and the
operator $\matmul$ denotes a matrix multiplication.
Neural nets usually have one or more fully connected layers, each of
which multiplies a matrix with a vector.
Some neural nets have convolution layers and such DNNs are also called
Convolutional Neural Nets or CNNs.
For the purpose of this paper, a convolution can be considered as a
(heavy) matrix-matrix multiplication. The size of matrices manipulated
by a convolution layer grows linearly with {\it window size}
(typically 9 or 25), the number of {\it output channels} (typically
16, 32, or 64), and the size of the matrix input to this layer.
Therefore, fully connected layers are lighter computation-wise compared
to convolution layers. However, the model size  of fully connected
layers is larger than those of convolution layers.
In general, DNNs are computationally heavy but provide
much better accuracies on computer vision tasks than the classifiers
discussed above.

\paragraph{State-of-the-art classifiers.}
Finally, there are a class of machine learning classifiers that are
much more efficient than
DNNs and provide reasonably good accuracies on standard learning
tasks. \bonsai~\cite{bonsai} is a state-of-the art classifier in this
class and \tool provides the first 2PC protocol for it.
\bonsai takes as input $x \in \R^d$, and its model consists of a
binary tree with $N$ nodes, and a matrix $Z$. Each node $j$ contains
matrices $W_j$ and $V_j$, and a vector $\theta_{j}$. The internal node
$j$  evaluates a predicate $(\theta_j^T \matmul Z \matmul x) > 0$ to decide whether
to pass $x$ to the left child $2j+1$ or the right child $2j+2$.
The predicted value is

%\aseem{I had a hard time parsing this. (1) Should we use $\cdot$ for
%  matrix multiplication as we did for DNNs? (2) Precedence of $\circ$
%  over scalar multiplication (or is it associative so doesn't matter?)
%(3) The argument to $f$ is a scalar (right?), and so is its output,
%so then is $\circ$ defined OR is it applying $f$ pointwise to a vector?}
%\nc{(1) I agree we should do that to be consistent with before. (2) I assumed $f()$ is computed to give a vector somehow and $I_j(x)W_j^TZx$ also gives a vector somehow and then we do $\circ$; I might be wrong. (3) Hmm.. I thought it was a vector, but I dont know how. (4) Can we specify somewhere the dimensions of these matrices?}
%\vspace{-10pt}
\[
{\sf argmax} \sum_{j = 0}^{N - 1} I_j(x)[ (W_j^T\matmul Z\matmul x) \circ (f(V_j^T\matmul Z\matmul x))] 
\]
Here, $I_j(x)$ is 1 if the $j^{th}$ node is present on the path traversed by $x$
and is zero otherwise. 
The operation $\circ$ is a pointwise multiplication of two vectors, $W_j$'s and $V_j$'s
are matrices of appropriate dimensions. The activation function $f$ is given by $f(y) = y$ if
$-1 < y < 1$ and $\mathrm{sign}(y)$ otherwise.
%% ($f$ on vectors and
%% matrices is defined point-wise on every element of the
%% vector/matrix).
%\bonsai can be
%seen as a variant of decision trees where the prediction is a function
%of the path traversed from root to the leaf and not just the leaf
%itself.

In the following, we implement these classifiers in \tool and report
the time taken for making secure predictions. Ideally, the machine
learning classifiers are mathematical expressions over $\R$ that are
usually approximated by floating-point operations. 
As is standard, we port the classifiers to integer manipulating
programs by scaling the models and rounding~\cite{minionn}. These
ported classifiers are then implemented in \tool.

\newpage

\section{Full formal definitions and auxiliary lemmas}
\label{app:formal}

\includepdf[pages=-]{formal.pdf}

\includepdf[pages=-]{proofs.pdf}


\end{document}


